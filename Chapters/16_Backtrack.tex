\chapter{Backtracking}
Dato un problema si dice soluzione ammissibile una soluzione che soddifa un insieme di criteri: nel caso dello zaino un sottoinsieme di oggetti di peso inferiore alla capacit\`a, mentre nella
sottosequenza comune una stringa che \`e sottosequenza di entrambe le stringhe in input. Nei problemi di ottimizazione viene definita una funzione di costo o guadagno definita 
sull'insieme delle soluzioni ammissibili: nello zaino \`e la somma dei guadagni degli oggetti selezionati, mentre nella sottosequenza comune massimale \`e la lunghezza della stringa.
\section{Brute force}
In alcuni problemi \`e richiesto o necessario esplorare l'intero spazio delle soluzioni ammissibili. Nell'enumerazione si richiede di elencare tutte le soluzioni possibili, nella ricerca
di trovare una soluzione ammissibile in uno spazio delle soluzioni molto grande, nel conteggio contare tutte le soluzioni ammissibili (quando non \`e possibile contarle in modo 
analitico), mentre nell'ottimizzazione si trovare una delle soluzioni ammissibili migliori rispetto ad un criterio di valutazione. 
\subsection{Costruire lo spazio delle soluzioni \`e costoso}
Lo spazio delle possibili soluzioni pu\`o essere superpolinomiale e a volta \`e l'unica strada possibile. A volte \`e possibile analizzare solo una parte dello spazio.
\section{Backtracking}
Il backtracking \`e una tecnica di programmazione che prova a fare qualcosa e se non va bene lo disfa e prova qualcos altro fino a che non arriva a una soluzione. Pu\`o essere 
implementata in maniera ricorsiva (un metodo sistematico per esplorare uno spazio di ricerca utilizzando la ricorsione per memorizzare le scelte fatte fin'ora) o iterativa (utilizzando
un approccio greedy, tornando eventualmente sui propri passi). Questa tecnica algoritmica deve essere naturalmente personalizzata per ogni applicazione individuale.
\subsection{Organizzazione generale}
Una soluzione viene rappresentata come un vettore $S[1\dots n]$ il contenuto degli elementi $S[i]$ \`e preso da un insieme di scelte $C$ dipendente dal problema.
\subsection{Soluzioni parziali}
Ad ogni passo si parte da una soluzione parziale $S[1\dots k]$ in cui $k\ge 0$ scelte sono state prese. Se $S[1\dots k]$ \`e una soluzione ammissibile viene processata, se non \`e 
una soluzione completa se possibile la si estende con una delle possibili scelte in $S[1\dots k+1]$, altrimenti si cancella l'elemento $S[k]$ (backtrack) e si riparte dalla soluzione
$S[1\dots k-1]$.
\subsection{Albero delle decisioni}
Durante l'algoritmo si genera un albero delle decisioni equivalente allo spazio di ricerca la cui radice \`e la soluzione parziale vuota, i nodi interni le soluzioni parziali e le
foglie le soluzioni ammissibili.
\subsection{Pruning}
I rami dell'albero che sicuramente non portano a soluzioni ammissibili possono essere potati o ``pruned" e la valutazione viene fatta nelle soluzioni parziali radici nel sottoalbero da
potare.
\section{Enumerazione}
\begin{algorithm}[H]
\DontPrintSemicolon
\SetKwComment{comment}{$\%$}{}
\SetKwRepeat{Do}{do}{while}
\SetKw{Int}{int}
\SetKw{Float}{float}
\SetKw{Boolean}{boolean}
\SetKw{New}{new}
\SetKw{True}{true}
\SetKw{False}{false}
\SetKw{Not}{not}
\SetKw{And}{and}
\SetKw{Or}{or}
\SetKw{Down}{down}
\SetKw{To}{to}
\SetKw{New}{new}
\SetKw{Return}{return}
\SetKw{Nil}{nil}
\SetKw{Print}{print}
\SetKw{Void}{void}

\SetKwData{Item}{Item}
\SetKwData{Tree}{Tree}
\SetKwData{PriorityQueue}{PriorityQueue}
\SetKwData{Mfset}{Mfset}
\SetKwData{Graph}{Graph}
\SetKwData{N}{n}
\SetKwData{Space}{ }
\SetKwData{Parent}{parent}
\SetKwData{Rank}{rank}
\SetKwData{Set}{Set}
\SetKwData{List}{List}
\SetKwData{F}{f}
\SetKwData{Left}{left}
\SetKwData{Right}{right}
\SetKwData{Mfset}{Mfset}
\SetKwData{PriorityItem}{PriorityItem}
\SetKwData{Node}{Node}

\SetKwFunction{Max}{max}
\SetKwFunction{Min}{min}
\SetKwFunction{Insert}{insert}
\SetKwFunction{ListCos}{List}
\SetKwFunction{Head}{head}
\SetKwFunction{Len}{len}
\SetKwFunction{SetCos}{Set}
\SetKwFunction{TreeCos}{Tree}
\SetKwFunction{MinPriorityQueue}{MinPriorityQueue}
\SetKwFunction{DeleteMin}{deleteMin}
\SetKwFunction{MfsetCos}{Mfset}
\SetKwFunction{Find}{find}
\SetKwFunction{Merge}{merge}
\SetKwFunction{IsEmpty}{isEmpty}
\SetKwFunction{Adj}{adj}
\SetKwFunction{Decrease}{decrease}


\SetKwProg{Fn}{}{}

\SetKwFunction{Enumeration}{enumeration}
\SetKwFunction{Choices}{choices}
\SetKwFunction{IsAdmissible}{isAdmissible}
\SetKwFunction{ProcessSolution}{processSolution}
\SetKwFunction{Accept}{accept}
\SetKwFunction{}{}

\caption{\protect\Enumeration{$<$dati problema$>$, \protect\Item[] S, \protect\Int i, $<$dati parziali$>$}}
\comment{Verifica se $S[1\dots i - 1]$ contiene una soluzione ammissibile}
\uIf{\Accept{$<$dati problema$>$, S, i $<$dati parziali$>$}}{
	\comment{Processa la soluzione}
	\ProcessSolution{$<$dati problema$>$, S, i, $<$dati parziali$>$}\;
}
\Else{
	\comment{Calcola l'insieme delle scelte in funzione di $S[1\dots i - 1]$}
	\Set C = \Choices{$<$dati problema$>$, S, i, $<$dati parziali$>$}\;
	\comment{Itera sull'insieme delle scelte}
	\ForEach{c $\in$ C}{
		S[i] = c\;
		\comment{Chiamata ricorsiva}
		\Enumeration{$<$dati problema$>$, S, n, i + 1, $<$dati parziali$>$}\;
	}
}
\Return \False\;
\end{algorithm}

\subsection{Sottoinsiemi}
Il problema consiste nell'elencare tutti i sottoinsiemi dell'insieme $\{1, \dots, n\}$.
\begin{multicols}{2}
\input{Pseudocodice/126_SubsetRec}
\columnbreak
\input{Pseudocodice/128_ProcessSolution}
\end{multicols}
Come richiesto dal problema tutto lo spazio possibile viene esplorato con complessit\`a $\Theta(n\cdot 2^n)$.
\subsubsection{Versione iterativa}
\begin{algorithm}[H]
\DontPrintSemicolon
\SetKwComment{comment}{$\%$}{}
\SetKwRepeat{Do}{do}{while}
\SetKw{Int}{int}
\SetKw{Float}{float}
\SetKw{Boolean}{boolean}
\SetKw{New}{new}
\SetKw{True}{true}
\SetKw{False}{false}
\SetKw{Not}{not}
\SetKw{And}{and}
\SetKw{Or}{or}
\SetKw{Down}{down}
\SetKw{To}{to}
\SetKw{New}{new}
\SetKw{Return}{return}
\SetKw{Nil}{nil}
\SetKw{Print}{print}
\SetKw{Println}{println}
\SetKw{Void}{void}

\SetKwData{Item}{Item}
\SetKwData{Tree}{Tree}
\SetKwData{PriorityQueue}{PriorityQueue}
\SetKwData{Mfset}{Mfset}
\SetKwData{Graph}{Graph}
\SetKwData{N}{n}
\SetKwData{Space}{ }
\SetKwData{Parent}{parent}
\SetKwData{Rank}{rank}
\SetKwData{Set}{Set}
\SetKwData{List}{List}
\SetKwData{F}{f}
\SetKwData{Left}{left}
\SetKwData{Right}{right}
\SetKwData{Mfset}{Mfset}
\SetKwData{PriorityItem}{PriorityItem}
\SetKwData{Node}{Node}

\SetKwFunction{Max}{max}
\SetKwFunction{Min}{min}
\SetKwFunction{Insert}{insert}
\SetKwFunction{ListCos}{List}
\SetKwFunction{Head}{head}
\SetKwFunction{Len}{len}
\SetKwFunction{SetCos}{Set}
\SetKwFunction{TreeCos}{Tree}
\SetKwFunction{MinPriorityQueue}{MinPriorityQueue}
\SetKwFunction{DeleteMin}{deleteMin}
\SetKwFunction{MfsetCos}{Mfset}
\SetKwFunction{Find}{find}
\SetKwFunction{Merge}{merge}
\SetKwFunction{IsEmpty}{isEmpty}
\SetKwFunction{Adj}{adj}
\SetKwFunction{Decrease}{decrease}


\SetKwProg{Fn}{}{}

\SetKwFunction{Subset}{subsets}
\SetKwFunction{Choices}{choices}
\SetKwFunction{IsAdmissible}{isAdmissible}
\SetKwFunction{ProcessSolution}{processSolution}
\SetKwFunction{Accept}{accept}
\SetKwFunction{SubsetRec}{subsetsRec}

\caption{\protect\Subset{\protect\Int n}}
\For{\Int j = 0 \To $2^n-1$}{
	\Print ``$\{$"\;
	\For{\Int i = 1 \To n - 1}{
		\If{(j \&\& $2^i$) $\neq$ 0}{ \comment{Bitwise and}
			\Print i, `` "\;
		}
	}
	\Println ``$\}$"\;
}
\end{algorithm}

Complessit\`a $\Theta(n\cdot 2^n)$.
\subsection{Permutazioni}
Stampa tutte le permutazioni di un insieme $A$.
\begin{multicols}{2}
\begin{algorithm}[H]
\DontPrintSemicolon
\SetKwComment{comment}{$\%$}{}
\SetKwRepeat{Do}{do}{while}
\SetKw{Int}{int}
\SetKw{Float}{float}
\SetKw{Boolean}{boolean}
\SetKw{New}{new}
\SetKw{True}{true}
\SetKw{False}{false}
\SetKw{Not}{not}
\SetKw{And}{and}
\SetKw{Or}{or}
\SetKw{Down}{down}
\SetKw{To}{to}
\SetKw{New}{new}
\SetKw{Return}{return}
\SetKw{Nil}{nil}
\SetKw{Print}{priint}
\SetKw{Println}{println}
\SetKw{Void}{void}

\SetKwData{Item}{Item}
\SetKwData{Tree}{Tree}
\SetKwData{PriorityQueue}{PriorityQueue}
\SetKwData{Mfset}{Mfset}
\SetKwData{Graph}{Graph}
\SetKwData{N}{n}
\SetKwData{Space}{ }
\SetKwData{Parent}{parent}
\SetKwData{Rank}{rank}
\SetKwData{Set}{Set}
\SetKwData{List}{List}
\SetKwData{F}{f}
\SetKwData{Left}{left}
\SetKwData{Right}{right}
\SetKwData{Mfset}{Mfset}
\SetKwData{PriorityItem}{PriorityItem}
\SetKwData{Node}{Node}

\SetKwFunction{Max}{max}
\SetKwFunction{Min}{min}
\SetKwFunction{Insert}{insert}
\SetKwFunction{ListCos}{List}
\SetKwFunction{Head}{head}
\SetKwFunction{Len}{len}
\SetKwFunction{SetCos}{Set}
\SetKwFunction{TreeCos}{Tree}
\SetKwFunction{MinPriorityQueue}{MinPriorityQueue}
\SetKwFunction{DeleteMin}{deleteMin}
\SetKwFunction{MfsetCos}{Mfset}
\SetKwFunction{Find}{find}
\SetKwFunction{Merge}{merge}
\SetKwFunction{IsEmpty}{isEmpty}
\SetKwFunction{Adj}{adj}
\SetKwFunction{Decrease}{decrease}
\SetKwFunction{Swap}{swap}


\SetKwProg{Fn}{}{}

\SetKwFunction{PermRec}{permRec}
\SetKwFunction{Choices}{choices}
\SetKwFunction{IsAdmissible}{isAdmissible}
\SetKwFunction{ProcessSolution}{processSolution}
\SetKwFunction{Accept}{accept}
\SetKwFunction{}{}

\caption{\protect\PermRec{\protect\Item[] S, \protect\Int i}}
\comment{Caso base con un carattere}
\uIf{i == 1}{
	\Println S\;
}
\Else{
	\For{\Int j = 1 \To i}{
		\Swap{S, i, j}\;
		\PermRec{S, i-1}\;
		\Swap{S, i, j}\;
	}
}
\end{algorithm}

\columnbreak
\input{Pseudocodice/131_Permutation}
Complessit\`a:
\begin{itemize}
	\item $n!$ permutazioni.
	\item $\Theta(n)$ per stamparle tutte.
	\item $2n$ swap per ogni permutazione.
	\item Costo totale: $\Theta(n\cdot n!)$.
\end{itemize}
\end{multicols}
\subsection{Enumerazione schema completo}
\begin{algorithm}[H]
\DontPrintSemicolon
\SetKwComment{comment}{$\%$}{}
\SetKwRepeat{Do}{do}{while}
\SetKw{Int}{int}
\SetKw{Float}{float}
\SetKw{Boolean}{boolean}
\SetKw{New}{new}
\SetKw{True}{true}
\SetKw{False}{false}
\SetKw{Not}{not}
\SetKw{And}{and}
\SetKw{Or}{or}
\SetKw{Down}{down}
\SetKw{To}{to}
\SetKw{New}{new}
\SetKw{Return}{return}
\SetKw{Nil}{nil}
\SetKw{Print}{print}
\SetKw{Void}{void}

\SetKwData{Item}{Item}
\SetKwData{Tree}{Tree}
\SetKwData{PriorityQueue}{PriorityQueue}
\SetKwData{Mfset}{Mfset}
\SetKwData{Graph}{Graph}
\SetKwData{N}{n}
\SetKwData{Space}{ }
\SetKwData{Parent}{parent}
\SetKwData{Rank}{rank}
\SetKwData{Set}{Set}
\SetKwData{List}{List}
\SetKwData{F}{f}
\SetKwData{Left}{left}
\SetKwData{Right}{right}
\SetKwData{Mfset}{Mfset}
\SetKwData{PriorityItem}{PriorityItem}
\SetKwData{Node}{Node}

\SetKwFunction{Max}{max}
\SetKwFunction{Min}{min}
\SetKwFunction{Insert}{insert}
\SetKwFunction{ListCos}{List}
\SetKwFunction{Head}{head}
\SetKwFunction{Len}{len}
\SetKwFunction{SetCos}{Set}
\SetKwFunction{TreeCos}{Tree}
\SetKwFunction{MinPriorityQueue}{MinPriorityQueue}
\SetKwFunction{DeleteMin}{deleteMin}
\SetKwFunction{MfsetCos}{Mfset}
\SetKwFunction{Find}{find}
\SetKwFunction{Merge}{merge}
\SetKwFunction{IsEmpty}{isEmpty}
\SetKwFunction{Adj}{adj}
\SetKwFunction{Decrease}{decrease}


\SetKwProg{Fn}{}{}

\SetKwFunction{Enumeration}{enumeration}
\SetKwFunction{Choices}{choices}
\SetKwFunction{IsAdmissible}{isAdmissible}
\SetKwFunction{ProcessSolution}{processSolution}
\SetKwFunction{Accept}{accept}
\SetKwFunction{Reject}{reject}

\caption{\protect\Enumeration{$<$dati problema$<$, \protect\Item[] S, \protect\Int i,$<$dati parziali$>$}}
\uIf{\Accept{$<$dati problema$>$, S, i $<$dati parziali$>$}}{
	\comment{Processa la soluzione $S$ in quanto ammissibile}
	\ProcessSolution{$<$dati problema$>$, S, i, $<$dati parziali$>$}\;
}
\uElseIf{\Reject{$<$dati problema$>$, S, i, $<$dati parziali$>$}}{
	\Return\;
}
\Else{
	\comment{Ricorsione}
	\Set C = \Choices{$<$dati problema$>$, S, i, $<$dati parziali$>$}\;
	\ForEach{c $\in$ C}{
		S[i] = c\;
		\Enumeration{$<$dati problema$>$, S, n, i + 1, $<$dati parziali$>$}\;
	}
}
\end{algorithm}

Ci si pu\`o interrompere alla prima soluzione introcendo un \emph{return} dopo \emph{processSolution($<$dati problema, S, i, $<$dati parziali$>$)}.
\subsection{K-sottoinsiemi}
Il problema consiste nell'elencare tutti i sottoinsiemi di $k$ elementi di un insieme $\{1, \dots, n\}$. Si pu\`o in questo caso specializzare l'algoritmo generico introducendo del
pruning migliorandone l'efficienza. Questo avviene introducendo la variabile \emph{missing} che permette di evitare di ricontare tutte le volte i bit a $1$ ed evitando di proseguire
in rami che non possono dare origine alla soluzione (non ci sono altri elementi da scegliere o quelli rimanenti sono troppi pochi).\\
\begin{algorithm}[H]
\DontPrintSemicolon
\SetKwComment{comment}{$\%$}{}
\SetKwRepeat{Do}{do}{while}
\SetKw{Int}{int}
\SetKw{Float}{float}
\SetKw{Boolean}{boolean}
\SetKw{New}{new}
\SetKw{True}{true}
\SetKw{False}{false}
\SetKw{Not}{not}
\SetKw{And}{and}
\SetKw{Or}{or}
\SetKw{Down}{down}
\SetKw{To}{to}
\SetKw{New}{new}
\SetKw{Return}{return}
\SetKw{Nil}{nil}
\SetKw{Print}{priint}
\SetKw{Println}{println}
\SetKw{Void}{void}

\SetKwData{Item}{Item}
\SetKwData{Tree}{Tree}
\SetKwData{PriorityQueue}{PriorityQueue}
\SetKwData{Mfset}{Mfset}
\SetKwData{Graph}{Graph}
\SetKwData{N}{n}
\SetKwData{Space}{ }
\SetKwData{Parent}{parent}
\SetKwData{Rank}{rank}
\SetKwData{Set}{Set}
\SetKwData{List}{List}
\SetKwData{F}{f}
\SetKwData{Left}{left}
\SetKwData{Right}{right}
\SetKwData{Mfset}{Mfset}
\SetKwData{PriorityItem}{PriorityItem}
\SetKwData{Node}{Node}

\SetKwFunction{Max}{max}
\SetKwFunction{Min}{min}
\SetKwFunction{Insert}{insert}
\SetKwFunction{ListCos}{List}
\SetKwFunction{Head}{head}
\SetKwFunction{Len}{len}
\SetKwFunction{SetCos}{Set}
\SetKwFunction{TreeCos}{Tree}
\SetKwFunction{MinPriorityQueue}{MinPriorityQueue}
\SetKwFunction{DeleteMin}{deleteMin}
\SetKwFunction{MfsetCos}{Mfset}
\SetKwFunction{Find}{find}
\SetKwFunction{Merge}{merge}
\SetKwFunction{IsEmpty}{isEmpty}
\SetKwFunction{Adj}{adj}
\SetKwFunction{Decrease}{decrease}
\SetKwFunction{Swap}{swap}


\SetKwProg{Fn}{}{}

\SetKwFunction{PermRec}{kssRec}
\SetKwFunction{Choices}{choices}
\SetKwFunction{IsAdmissible}{isAdmissible}
\SetKwFunction{ProcessSolution}{processSolution}
\SetKwFunction{Accept}{accept}
\SetKwFunction{}{}

\caption{\protect\PermRec{\protect\Int n, \protect\Int missing, \protect\Item[] S, \protect\Int i}}
\uIf{missing == 0}{
	\ProcessSolution{S, i - 1}\;
}
\ElseIf{i $\le$ n \And 0 $<$ missing $\le$ n - (i - 1)}{
	\ForEach{c $\in$ $\{0, 1\}$}{
		S[i] = c\;
		\PermRec{n, missing - c, S, i + 1}\;
	}
}
\end{algorithm}

\subsection{Somma di sottoinsiemi}
Dati un insieme $A = \{a_1, \dots, a_n\}$ di interi positivi e un intero positivo $k$, si trovi un sottoinsieme $S$ di indici in $\{1, \dots, n\}$ tale che 
$\sum\limits_{i\in S}a_i = k$. Lo si risolve in backtracking in tempo $O(2^n)$ e pu\`o essere risolto tramite programmazione dinamica in tempo $O(kn)$, pseudopolinomiale. Si interrompe
l'esecuzione alla prima soluzione trovata.\\
\begin{algorithm}[H]
\DontPrintSemicolon
\SetKwComment{comment}{$\%$}{}
\SetKwRepeat{Do}{do}{while}
\SetKw{Int}{int}
\SetKw{Float}{float}
\SetKw{Boolean}{boolean}
\SetKw{New}{new}
\SetKw{True}{true}
\SetKw{False}{false}
\SetKw{Not}{not}
\SetKw{And}{and}
\SetKw{Or}{or}
\SetKw{Down}{down}
\SetKw{To}{to}
\SetKw{New}{new}
\SetKw{Return}{return}
\SetKw{Nil}{nil}
\SetKw{Print}{priint}
\SetKw{Println}{println}
\SetKw{Void}{void}

\SetKwData{Item}{Item}
\SetKwData{Tree}{Tree}
\SetKwData{PriorityQueue}{PriorityQueue}
\SetKwData{Mfset}{Mfset}
\SetKwData{Graph}{Graph}
\SetKwData{N}{n}
\SetKwData{Space}{ }
\SetKwData{Parent}{parent}
\SetKwData{Rank}{rank}
\SetKwData{Set}{Set}
\SetKwData{List}{List}
\SetKwData{F}{f}
\SetKwData{Left}{left}
\SetKwData{Right}{right}
\SetKwData{Mfset}{Mfset}
\SetKwData{PriorityItem}{PriorityItem}
\SetKwData{Node}{Node}

\SetKwFunction{Max}{max}
\SetKwFunction{Min}{min}
\SetKwFunction{Insert}{insert}
\SetKwFunction{ListCos}{List}
\SetKwFunction{Head}{head}
\SetKwFunction{Len}{len}
\SetKwFunction{SetCos}{Set}
\SetKwFunction{TreeCos}{Tree}
\SetKwFunction{MinPriorityQueue}{MinPriorityQueue}
\SetKwFunction{DeleteMin}{deleteMin}
\SetKwFunction{MfsetCos}{Mfset}
\SetKwFunction{Find}{find}
\SetKwFunction{Merge}{merge}
\SetKwFunction{IsEmpty}{isEmpty}
\SetKwFunction{Adj}{adj}
\SetKwFunction{Decrease}{decrease}
\SetKwFunction{Swap}{swap}


\SetKwProg{Fn}{}{}

\SetKwFunction{PermRec}{ssRec}
\SetKwFunction{Choices}{choices}
\SetKwFunction{IsAdmissible}{isAdmissible}
\SetKwFunction{ProcessSolution}{processSolution}
\SetKwFunction{Accept}{accept}
\SetKwFunction{SubsetSum}{subsetSum}

\caption{Suset sum}
\Fn{\SubsetSum{\Int[] A, \Int n, \Int k}}{
	\Int[] S = \New \Int$[1\dots n]$\;
	\PermRec{A, n, k, S, 1}\;
}

\Boolean\Space\Fn{\PermRec{\Int[] A, \Int n, \Int missing, \Item[] S, \Int i}}{
	\uIf{missing == 0}{
		\ProcessSolution{S, i - 1}\comment{Stampa gli indici della soluzione}
		\Return \True\;
	}
	\uElseIf{i $>$ n \Or missing $<$ 0}{
		\Return \False \comment{Terminati i valori o somma eccessiva}
	}
	\Else{
		\ForEach{c $\in$ $\{0, 1\}$}{
			S[i] = c\;
			\If{\PermRec{A, n, missing - A[i] $\cdot$ c, S, i + 1}}{
				\Return \True\;
			}
		}
		\Return \False\;
	}
}
\end{algorithm}

\section{Problemi}
\subsection{Problema delle otto regine}
Il problema consiste di posizionare $n$ nergine in una scacchiera $n\times n$ in modo tale che nessuna regini ne minacci un'altra.
\subsubsection{Ci sono $\mathbf{n^2}$ caselle dove piazzare una regina}
\begin{tabular}{|c|c|}
	\hline
	$S[1\dots n^2]$ array binario & $S[i]=$\textbf{true} $\Rightarrow$ regina in $S[i]$ \\
	\hline
	Controllo soluzione & se $i = n^2$ \\
	\hline
	\emph{choices(S, n, i)} & $\{$\textbf{true}, \textbf{false}$\}$\\
	\hline
	pruning & \makecell{Se la nuova regina minaccia una delle \\regine esistenti restituisce $\emptyset$}\\
	\hline
	$\#$ soluzioni per $n=8$ & $2^{64}\approx 1.84\cdot 10^{19}$\\
	\hline
\end{tabular}\\
La matrice binaria \`e molto sparsa.
\subsubsection{Si devono piazzare $\mathbf{n}$ regine in $\mathbf{n^2}$ caselle}
\begin{tabular}{|c|c|}
	\hline
	$S[1\dots n]$ coordinate in $\{1\dots n^2\}$ & $S[i]$ coordinata della regina $i$\\
	\hline
	Controllo soluzione & se $i = n$ \\
	\hline
	\emph{choices(S, n, i)} & $\{1\dots n^2\}$\\
	\hline
	pruning & \makecell{Restituisce il sottoinsieme\\ delle mosse legali}\\
	\hline
	$\#$ soluzioni per $n=8$ & $(n^2)^n = 64^{8}\approx 2.81\cdot 10^{14}$\\
	\hline
\end{tabular}\\
Si nota un miglioramento ma lo spazio \`e ancora grande, inoltre \`e difficile distinguere le permutazioni di una soluzione.
\subsubsection{Non mettere regine in caselle precedenti a quelle gi\`a scelte}
\begin{tabular}{|c|c|}
	\hline
	$S[1\dots n]$ coordinate in $\{1\dots n^2\}$ & $S[i]$ coordinata della regina $i$\\
	\hline
	Controllo soluzione & se $i = n$ \\
	\hline
	\emph{choices(S, n, i)} & $\{1\dots n^2\}$\\
	\hline
	pruning & \makecell{Restituisce il sottoinsieme\\ delle mosse legali, $S[i]>S[i-1]$}\\
	\hline
	$\#$ soluzioni per $n=8$ & $\dfrac{(n^2)^n}{n!} = \dfrac{2^{48}}{40320}\approx 6.98\cdot 10^{9}$\\
	\hline
\end{tabular}
\subsubsection{Ogni riga della scacchiera deve contenere esattamente una regina}
\begin{tabular}{|c|c|}
	\hline
	$S[1\dots n]$ coordinate in $\{1\dots n\}$ & $S[i]$ colonna della regina $i$, dove riga $= i$\\
	\hline
	Controllo soluzione & se $i = n$ \\
	\hline
	\emph{choices(S, n, i)} & $\{1\dots n\}$\\
	\hline
	pruning & Restituisce le colonne legali\\
	\hline
	$\#$ soluzioni per $n=8$ & $n^n = 8^8\approx 1.67\cdot 10^{7}$\\
	\hline
\end{tabular}
\subsubsection{Anche ogni colonna deve contenere esattamente una regina}
\begin{tabular}{|c|c|}
	\hline
	$S[1\dots n]$ coordinate in $\{1\dots n\}$ & Permutazione di $\{1\dots n\}$\\
	\hline
	Controllo soluzione & se $i = n$ \\
	\hline
	\emph{choices(S, n, i)} & $\{1\dots n\}$\\
	\hline
	pruning & Elimina le diagonali\\
	\hline
	$\#$ soluzioni per $n=8$ & $n! = 8! = 4.320$\\
	\hline
\end{tabular}\\
Le soluzioni effettivamente visitate sono $15720$.
\subsubsection{Minimum-conflicts heuristic}
Si parte da una soluzione iniziale ragionevolmente buona e si muove il pezzo con il pi\`u grande numero di conflitti nella casella della stessa colonna che genera il numero minimo di 
conflitti e si ripete fino a quando non ci sono pi\`u pezzi da muovere. \`E in tempo lineare ma non garantisce che la terminazione sia sempre corretta.\\
\begin{algorithm}[H]
\DontPrintSemicolon
\SetKwComment{comment}{$\%$}{}
\SetKwRepeat{Do}{do}{while}
\SetKw{Int}{int}
\SetKw{Float}{float}
\SetKw{Boolean}{boolean}
\SetKw{New}{new}
\SetKw{True}{true}
\SetKw{False}{false}
\SetKw{Not}{not}
\SetKw{And}{and}
\SetKw{Or}{or}
\SetKw{Down}{down}
\SetKw{To}{to}
\SetKw{New}{new}
\SetKw{Return}{return}
\SetKw{Nil}{nil}
\SetKw{Print}{print}
\SetKw{Void}{void}

\SetKwData{Item}{Item}
\SetKwData{Tree}{Tree}
\SetKwData{PriorityQueue}{PriorityQueue}
\SetKwData{Mfset}{Mfset}
\SetKwData{Graph}{Graph}
\SetKwData{N}{n}
\SetKwData{Space}{ }
\SetKwData{Parent}{parent}
\SetKwData{Rank}{rank}
\SetKwData{Set}{Set}
\SetKwData{List}{List}
\SetKwData{F}{f}
\SetKwData{Left}{left}
\SetKwData{Right}{right}
\SetKwData{Mfset}{Mfset}
\SetKwData{PriorityItem}{PriorityItem}
\SetKwData{Node}{Node}

\SetKwFunction{Max}{max}
\SetKwFunction{Min}{min}
\SetKwFunction{Insert}{insert}
\SetKwFunction{ListCos}{List}
\SetKwFunction{Head}{head}
\SetKwFunction{Len}{len}
\SetKwFunction{SetCos}{Set}
\SetKwFunction{TreeCos}{Tree}
\SetKwFunction{MinPriorityQueue}{MinPriorityQueue}
\SetKwFunction{DeleteMin}{deleteMin}
\SetKwFunction{MfsetCos}{Mfset}
\SetKwFunction{Find}{find}
\SetKwFunction{Merge}{merge}
\SetKwFunction{IsEmpty}{isEmpty}
\SetKwFunction{Adj}{adj}
\SetKwFunction{Decrease}{decrease}


\SetKwProg{Fn}{}{}

\SetKwFunction{Queens}{queens}
\SetKwFunction{Choices}{mosse}
\SetKwFunction{IsAdmissible}{isAdmissible}
\SetKwFunction{ProcessSolution}{processSolution}
\SetKwFunction{}{}
\SetKwFunction{}{}

\caption{\protect\Queens{\protect\Int n, \protect\Int[] S, \protect\Int i}}
\uIf{i $>$ n}{
	\Print S\;
}
\Else{
	\For{\Int j = 1 \To n}{\comment{Prova a piazzare la regina nella colonna $j$}
		\Boolean legal = \True\;
		\For{\Int k = 1 \To i - 1}{
			\comment{Verifica le regine precedenti}
			\If{S[k] == j \Or S[k] == j + 1 - k \Or S[k] == j - i + k}{
				legal = \False\;
			}
		}
		\If{legal}{
			S[i] = j\;
			\Queens{n, S, i + 1}\;
		}
	}
}
\end{algorithm}

\subsection{Giro di cavallo}
Si consideri una scacchera $n\times n$, lo scopo \`e trovare un giro di cavallo, un percorso di mosse valide del cavallo in modo che ogni casella venga visitata al pi\`u una volta.
\paragraph{Soluzione}
Matrice $n\times n$ le cui celle contengono $0$ se la cella non \`e mai stata visitata e $i$ se la cella \`e stata visitata al passo $i$-esimo. Il numero di soluzioni \`e 
$64!\approx 10^{89}$ ma ad ogni passo si hanno al massimo $8$ caselle possibili e pertanto ne visito al pi\`u $8^{64}\approx 10^{57}$.\\
\begin{algorithm}[H]
\DontPrintSemicolon
\SetKwComment{comment}{$\%$}{}
\SetKwRepeat{Do}{do}{while}
\SetKw{Int}{int}
\SetKw{Float}{float}
\SetKw{Boolean}{boolean}
\SetKw{New}{new}
\SetKw{True}{true}
\SetKw{False}{false}
\SetKw{Not}{not}
\SetKw{And}{and}
\SetKw{Or}{or}
\SetKw{Down}{down}
\SetKw{To}{to}
\SetKw{New}{new}
\SetKw{Return}{return}
\SetKw{Nil}{nil}
\SetKw{Print}{print}
\SetKw{Void}{void}

\SetKwData{Item}{Item}
\SetKwData{Tree}{Tree}
\SetKwData{PriorityQueue}{PriorityQueue}
\SetKwData{Mfset}{Mfset}
\SetKwData{Graph}{Graph}
\SetKwData{N}{n}
\SetKwData{Space}{ }
\SetKwData{Parent}{parent}
\SetKwData{Rank}{rank}
\SetKwData{Set}{Set}
\SetKwData{List}{List}
\SetKwData{F}{f}
\SetKwData{Left}{left}
\SetKwData{Right}{right}
\SetKwData{Mfset}{Mfset}
\SetKwData{PriorityItem}{PriorityItem}
\SetKwData{Node}{Node}

\SetKwFunction{Max}{max}
\SetKwFunction{Min}{min}
\SetKwFunction{Insert}{insert}
\SetKwFunction{ListCos}{List}
\SetKwFunction{Head}{head}
\SetKwFunction{Len}{len}
\SetKwFunction{SetCos}{Set}
\SetKwFunction{TreeCos}{Tree}
\SetKwFunction{MinPriorityQueue}{MinPriorityQueue}
\SetKwFunction{DeleteMin}{deleteMin}
\SetKwFunction{MfsetCos}{Mfset}
\SetKwFunction{Find}{find}
\SetKwFunction{Merge}{merge}
\SetKwFunction{IsEmpty}{isEmpty}
\SetKwFunction{Adj}{adj}
\SetKwFunction{Decrease}{decrease}


\SetKwProg{Fn}{}{}

\SetKwFunction{Cavallo}{KnightTour}
\SetKwFunction{Choices}{moves}
\SetKwFunction{IsAdmissible}{isAdmissible}
\SetKwFunction{ProcessSolution}{processSolution}
\SetKwFunction{}{}
\SetKwFunction{}{}

\caption{\protect\Boolean \protect\Cavallo{\protect\Int[][] S, \protect\Int i, \protect\Int x, \protect\Int y}}
\comment{Se $i=64$ ho fatto $63$ mosse e ho completato un tour aperto}
\If{i == 64}{
	\ProcessSolution{S}\;
	\Return \True\;
}
\Set C = \Choices{S, x, y}\;
\ForEach{c $\in$ C}{
	S[x][y] = i\;
	\If{\Cavallo{S, i + 1, x + $m_x$[c], y + $m_y$[c]}}{
		\Return \True\;
	}
	S[x][y] = 0\;
}
\Return \False\;
\end{algorithm}



\begin{algorithm}[H]
\DontPrintSemicolon
\SetKwComment{comment}{$\%$}{}
\SetKwRepeat{Do}{do}{while}
\SetKw{Int}{int}
\SetKw{Float}{float}
\SetKw{Boolean}{boolean}
\SetKw{New}{new}
\SetKw{True}{true}
\SetKw{False}{false}
\SetKw{Not}{not}
\SetKw{And}{and}
\SetKw{Or}{or}
\SetKw{Down}{down}
\SetKw{To}{to}
\SetKw{New}{new}
\SetKw{Return}{return}
\SetKw{Nil}{nil}
\SetKw{Print}{print}
\SetKw{Void}{void}

\SetKwData{Item}{Item}
\SetKwData{Tree}{Tree}
\SetKwData{PriorityQueue}{PriorityQueue}
\SetKwData{Mfset}{Mfset}
\SetKwData{Graph}{Graph}
\SetKwData{N}{n}
\SetKwData{Space}{ }
\SetKwData{Parent}{parent}
\SetKwData{Rank}{rank}
\SetKwData{Set}{Set}
\SetKwData{List}{List}
\SetKwData{F}{f}
\SetKwData{Left}{left}
\SetKwData{Right}{right}
\SetKwData{Mfset}{Mfset}
\SetKwData{PriorityItem}{PriorityItem}
\SetKwData{Node}{Node}

\SetKwFunction{Max}{max}
\SetKwFunction{Min}{min}
\SetKwFunction{Insert}{insert}
\SetKwFunction{ListCos}{List}
\SetKwFunction{Head}{head}
\SetKwFunction{Len}{len}
\SetKwFunction{SetCos}{Set}
\SetKwFunction{TreeCos}{Tree}
\SetKwFunction{MinPriorityQueue}{MinPriorityQueue}
\SetKwFunction{DeleteMin}{deleteMin}
\SetKwFunction{MfsetCos}{Mfset}
\SetKwFunction{Find}{find}
\SetKwFunction{Merge}{merge}
\SetKwFunction{IsEmpty}{isEmpty}
\SetKwFunction{Adj}{adj}
\SetKwFunction{Decrease}{decrease}


\SetKwProg{Fn}{}{}

\SetKwFunction{}{}
\SetKwFunction{Choices}{moves}
\SetKwFunction{IsAdmissible}{isAdmissible}
\SetKwFunction{ProcessSolution}{processSolution}
\SetKwFunction{}{}
\SetKwFunction{}{}

\caption{\protect\Set \protect\Choices{\protect\Int[][] S, \protect\Int x, \protect\Int y}}
\Set C = \SetCos{}\;
\For{\Int i = 1 \To 8}{
	$n_x = x + m_x[i]$\;
	$n_y = y + m_y[i]$\;
	\If{$1 \le n_x \le 8$ \And $1\le n_y \le 8$ \And $S[n_x][n_y]==0$}{
		C.\Insert{i}\;
	}
}
\Return C\;
\end{algorithm}

$m_x = \{-1, +1, +2, +2, +1, -1, -2, -2\}$\\
$m_y = \{-2, -2, -1, +1, +2, +2, +1, -1\}$
\newpage
\subsection{Sudoku}
La soluzione proposta \`e molto veloce per $n=9$, ma \`e possibile generalizzare per $n=k^2$ ed esistono tecniche euristiche per fissare altri numeri che possono risolvere completamente
il problema o essere usate come pre-processamento.
\begin{multicols}{2}
\begin{algorithm}[H]
\DontPrintSemicolon
\SetKwComment{comment}{$\%$}{}
\SetKwRepeat{Do}{do}{while}
\SetKw{Int}{int}
\SetKw{Float}{float}
\SetKw{Boolean}{boolean}
\SetKw{New}{new}
\SetKw{True}{true}
\SetKw{False}{false}
\SetKw{Not}{not}
\SetKw{And}{and}
\SetKw{Or}{or}
\SetKw{Down}{down}
\SetKw{To}{to}
\SetKw{New}{new}
\SetKw{Return}{return}
\SetKw{Nil}{nil}
\SetKw{Print}{print}
\SetKw{Void}{void}

\SetKwData{Item}{Item}
\SetKwData{Tree}{Tree}
\SetKwData{PriorityQueue}{PriorityQueue}
\SetKwData{Mfset}{Mfset}
\SetKwData{Graph}{Graph}
\SetKwData{N}{n}
\SetKwData{Space}{ }
\SetKwData{Parent}{parent}
\SetKwData{Rank}{rank}
\SetKwData{Set}{Set}
\SetKwData{List}{List}
\SetKwData{F}{f}
\SetKwData{Left}{left}
\SetKwData{Right}{right}
\SetKwData{Mfset}{Mfset}
\SetKwData{PriorityItem}{PriorityItem}
\SetKwData{Node}{Node}

\SetKwFunction{Max}{max}
\SetKwFunction{Min}{min}
\SetKwFunction{Insert}{insert}
\SetKwFunction{ListCos}{List}
\SetKwFunction{Head}{head}
\SetKwFunction{Len}{len}
\SetKwFunction{SetCos}{Set}
\SetKwFunction{TreeCos}{Tree}
\SetKwFunction{MinPriorityQueue}{MinPriorityQueue}
\SetKwFunction{DeleteMin}{deleteMin}
\SetKwFunction{MfsetCos}{Mfset}
\SetKwFunction{Find}{find}
\SetKwFunction{Merge}{merge}
\SetKwFunction{IsEmpty}{isEmpty}
\SetKwFunction{Adj}{adj}
\SetKwFunction{Decrease}{decrease}


\SetKwProg{Fn}{}{}

\SetKwFunction{Sudoku}{sudoku}
\SetKwFunction{Choices}{moves}
\SetKwFunction{IsAdmissible}{isAdmissible}
\SetKwFunction{ProcessSolution}{processSolution}
\SetKwFunction{}{}
\SetKwFunction{}{}

\caption{\protect\Boolean \protect\Sudoku{\protect\Int[][] S, \protect\Int i}}
\If{i == 81}{
	\ProcessSolution{S, n}\;
	\Return \True\;
}
\Int x = i $\mod 9$\;
\Int y = $\lfloor\frac{i}{9}\rfloor$\;
\Set C = \Choices{S, x, y}\;
\Int old = S[x][y]\;
\ForEach{c $\in$ C}{
	S[x][y] = c\;
	\If{\Sudoku{S, i + 1}}{
		\Return \True\;
	}
}
S[x][y] = old\;
\Return \False\;
\end{algorithm}

\columnbreak
\begin{algorithm}[H]
\DontPrintSemicolon
\SetKwComment{comment}{$\%$}{}
\SetKwRepeat{Do}{do}{while}
\SetKw{Int}{int}
\SetKw{Float}{float}
\SetKw{Boolean}{boolean}
\SetKw{New}{new}
\SetKw{True}{true}
\SetKw{False}{false}
\SetKw{Not}{not}
\SetKw{And}{and}
\SetKw{Or}{or}
\SetKw{Down}{down}
\SetKw{To}{to}
\SetKw{New}{new}
\SetKw{Return}{return}
\SetKw{Nil}{nil}
\SetKw{Print}{print}
\SetKw{Void}{void}

\SetKwData{Item}{Item}
\SetKwData{Tree}{Tree}
\SetKwData{PriorityQueue}{PriorityQueue}
\SetKwData{Mfset}{Mfset}
\SetKwData{Graph}{Graph}
\SetKwData{N}{n}
\SetKwData{Space}{ }
\SetKwData{Parent}{parent}
\SetKwData{Rank}{rank}
\SetKwData{Set}{Set}
\SetKwData{List}{List}
\SetKwData{F}{f}
\SetKwData{Left}{left}
\SetKwData{Right}{right}
\SetKwData{Mfset}{Mfset}
\SetKwData{PriorityItem}{PriorityItem}
\SetKwData{Node}{Node}

\SetKwFunction{Max}{max}
\SetKwFunction{Min}{min}
\SetKwFunction{Insert}{insert}
\SetKwFunction{ListCos}{List}
\SetKwFunction{Head}{head}
\SetKwFunction{Len}{len}
\SetKwFunction{SetCos}{Set}
\SetKwFunction{TreeCos}{Tree}
\SetKwFunction{MinPriorityQueue}{MinPriorityQueue}
\SetKwFunction{DeleteMin}{deleteMin}
\SetKwFunction{MfsetCos}{Mfset}
\SetKwFunction{Find}{find}
\SetKwFunction{Merge}{merge}
\SetKwFunction{IsEmpty}{isEmpty}
\SetKwFunction{Adj}{adj}
\SetKwFunction{Decrease}{decrease}


\SetKwProg{Fn}{}{}

\SetKwFunction{}{}
\SetKwFunction{Choices}{moves}
\SetKwFunction{IsAdmissible}{isAdmissible}
\SetKwFunction{ProcessSolution}{processSolution}
\SetKwFunction{Check}{check}
\SetKwFunction{}{}

\caption{\protect\Set \protect\Choices{\protect\Int[][] S, \protect\Int x, \protect\Int y}}
\Set C = \SetCos{}\;
\uIf{S[x][y] $\neq$ 0}{
	\comment{numero pre-inserito}
	C.\Insert{S[x][y]}\;
}
\Else{
	\comment{Verifica conflitti}
	\For{\Int c = 1 \To 9}{
		\If{\Check{S, x, y, c}}{
			C.\Insert{c}\;
		}
	}
}
\Return C\;
\end{algorithm}

\end{multicols}
\begin{algorithm}[H]
\DontPrintSemicolon
\SetKwComment{comment}{$\%$}{}
\SetKwRepeat{Do}{do}{while}
\SetKw{Int}{int}
\SetKw{Float}{float}
\SetKw{Boolean}{boolean}
\SetKw{New}{new}
\SetKw{True}{true}
\SetKw{False}{false}
\SetKw{Not}{not}
\SetKw{And}{and}
\SetKw{Or}{or}
\SetKw{Down}{down}
\SetKw{To}{to}
\SetKw{New}{new}
\SetKw{Return}{return}
\SetKw{Nil}{nil}
\SetKw{Print}{print}
\SetKw{Void}{void}

\SetKwData{Item}{Item}
\SetKwData{Tree}{Tree}
\SetKwData{PriorityQueue}{PriorityQueue}
\SetKwData{Mfset}{Mfset}
\SetKwData{Graph}{Graph}
\SetKwData{N}{n}
\SetKwData{Space}{ }
\SetKwData{Parent}{parent}
\SetKwData{Rank}{rank}
\SetKwData{Set}{Set}
\SetKwData{List}{List}
\SetKwData{F}{f}
\SetKwData{Left}{left}
\SetKwData{Right}{right}
\SetKwData{Mfset}{Mfset}
\SetKwData{PriorityItem}{PriorityItem}
\SetKwData{Node}{Node}

\SetKwFunction{Max}{max}
\SetKwFunction{Min}{min}
\SetKwFunction{Insert}{insert}
\SetKwFunction{ListCos}{List}
\SetKwFunction{Head}{head}
\SetKwFunction{Len}{len}
\SetKwFunction{SetCos}{Set}
\SetKwFunction{TreeCos}{Tree}
\SetKwFunction{MinPriorityQueue}{MinPriorityQueue}
\SetKwFunction{DeleteMin}{deleteMin}
\SetKwFunction{MfsetCos}{Mfset}
\SetKwFunction{Find}{find}
\SetKwFunction{Merge}{merge}
\SetKwFunction{IsEmpty}{isEmpty}
\SetKwFunction{Adj}{adj}
\SetKwFunction{Decrease}{decrease}


\SetKwProg{Fn}{}{}

\SetKwFunction{}{}
\SetKwFunction{Choices}{moves}
\SetKwFunction{IsAdmissible}{isAdmissible}
\SetKwFunction{ProcessSolution}{processSolution}
\SetKwFunction{Check}{check}
\SetKwFunction{}{}

\caption{\protect\Boolean \protect\Check{\protect\Int[][] S, \protect\Int x, \protect\Int y, \protect\Int c}}
\For{\Int j = 0 \To 8}{
	\If{S[x][j] == c}{
		\Return \False \comment{Controllo sulla colonna}
	}
	\If{S[j][y] == c}{
		\Return \False \comment{Controllo sulla riga}
	}
}
\Int $b_x$ = $\lfloor\frac{x}{3}\rfloor$\;
\Int $b_y$ = $\lfloor\frac{y}{3}\rfloor$\;
\For{\Int $i_x$ = 0 \To 2}{
	\For{\Int $i_y$ = 0 \To 2}{
		\If{S[$b_x\ \cdot\ 3\ +\ i_x$][$b_y\ \cdot\ 3\ +\ i_y$] == c}{
			\Return \False \comment{Controllo sulla sottotabella}
		}
	}
}
\Return \True\;
\end{algorithm}


