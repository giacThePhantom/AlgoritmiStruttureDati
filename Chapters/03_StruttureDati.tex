\chapter{Strutture dati}
Si intende per dato in un linguaggio di programmazione un valore che una variabile pu\`o assumere. Un tipo di dato astratto \`e una collezione di valori e un
insieme di operazioni ammesse su quei valori. Si dicono primitivi i tipi di dati forniti direttamente dal linguaggio di programmazione. La definizione di un
tipo di dato astratto si divide nella specifica, una descrizione di alto livello di come pu\`o essere utilizzata e nell'implementazione, la realizzazione 
vera e propria di basso livello. Le strutture dati sono collezioni di dati, caratterizzate dall'organizzazione della collezione stessa pi\`u che dal tipo di
dati contenuti. Le strutture dati sono caratterizzate da un insieme di operazioni che permettono di manipolarne la struttura e un modo sistematico per 
organizzare l'insieme di dati. Le strutture dati si dividono in:
\begin{itemize}
\item Lineari o non lineari rispetto alla presenza o meno di una sequenza.
\item Statiche o dinamiche se la dimensione della struttura pu\`o variare o meno.
\item Omogenee o disomogenee se possono contenere uno o diversi tipi di dati.
\end{itemize}
\section{Sequenza}
Una sequenza \`e una struttura dati dinamica, lineare che rappresenta una sequenza ordinata di valori, dove un valore pu\`o comparire pi\`u di una volta. \`E
importante l'ordine all'interno della sequenza. 
\subsubsection{Operazioni ammesse}
\begin{itemize}
\item Data una posizione \`e possibile aggiungere o eliminare un elemento in quella posizione. Si considerano anche per comodit\`a le posizioni $pos_0$ e 
$pos_{n+1}$.
\item \`E possibile accedere direttamente alla testa e alla coda della sequenza.
\item \`E possibile accedere sequenzialmente a tutti gli altri elementi.
\end{itemize}
\subsubsection{Specifica}
\begin{algorithm}[H]
\DontPrintSemicolon
\SetKwData{Item}{Item}
\SetKwData{Pos}{Pos}
\SetKwComment{comment}{$\%$}{}
\SetKwFunction{Isempty}{isEmpty}
\SetKwFunction{Finished}{finished}
\SetKwFunction{Head}{head}
\SetKwFunction{Tail}{tail}
\SetKwFunction{Next}{next}
\SetKwFunction{Prev}{prev}
\SetKwFunction{Insert}{insert}
\SetKwFunction{Remove}{remove}
\SetKwFunction{Read}{read}
\SetKwFunction{Write}{write}
\SetKwData{Sequence}{Sequence}

\caption{\protect\Sequence}

\comment{Restituisce True se la sequenza \`e vuota}
\textbf{boolean} \Isempty{}\;
\BlankLine
\comment{Restituisce True se $p$ \`e uguale a $pos_0$ o a $pos_{n+1}$}
\BlankLine
\textbf{boolean} \Finished{\Pos $p$}\;
\BlankLine
\comment{Restituisce la posizione del primo elemento}
\BlankLine
\Pos \Head{}\;
\BlankLine
\comment{Restituisce la posizione dell'ultimo elemento}
\BlankLine
\Pos \Tail{}\;
\BlankLine
\comment{Restituisce la posizione dell'elemento che segue $p$}
\BlankLine
\Pos \Next{\Pos $p$}\;
\BlankLine
\comment{Restituisce la posizione dell'elemento che precede $p$}
\BlankLine
\Pos \Prev{\Pos $p$}\;
\BlankLine
\comment{Inserisce l'elemento $v$ di tipo \Item nella posizione $p$}
\comment{Restituisce la posizione del nuovo elemento che diventa predecessore di $p$}
\BlankLine
\Pos \Insert{\Pos $p$, \Item $v$}\;
\BlankLine
\comment{Rimuove l'elemento alla posizione $p$}
\comment{Restituisce la posizione del successore di $p$}
\comment{Che diventa il successore del predecessore di $p$}
\BlankLine
\Pos \Remove{Pos $p$}\;
\BlankLine
\comment{Legge l'elemento di tipo \Item contenuto nella posizione $p$}
\BlankLine
\Item \Read{\Pos $p$}\;
\BlankLine
\comment{Scrive l'elemento $v$ di tipo \Item nella posizione $p$}
\BlankLine
\Write{\Pos $p$, \Item $v$}\;
\end{algorithm}

\newpage
\section{Insiemi}
Per insieme si intende una struttura dati dinamica, non lineare che memorizza una collezione non ordinata di elementi senza valori ripetuti. L'ordinamento
tra i valori sar\`a dato dall'eventuale relazione d'ordine definita sul tipo di elementi stessi. 
\subsubsection{Operazioni ammesse}
\begin{multicols}{2}
\begin{itemize}
\item Operazioni base:
\begin{itemize}
\item Inserimento.
\item Cancellazione.
\item Verifica contenimento.
\end{itemize}
\item Operazioni di ordinamento:
\begin{itemize}
\item Massimo.
\item Minimo.
\end{itemize}
\item Operazioni insiemistiche:
\begin{itemize}
\item Unione.
\item Intersezione.
\item Differenza.
\end{itemize}
\item Iteratori.
\begin{itemize}
\item \textbf{foreach} $x\in S$ \textbf{do}.
\end{itemize}
\end{itemize}
\end{multicols}
\subsubsection{Specifica}
\begin{algorithm}
\DontPrintSemicolon
\SetKwData{Item}{Item}
\SetKwData{Set}{Set}
\SetKwData{Int}{int}
\SetKw{Boolean}{boolean}
\SetKwComment{comment}{$\%$}{}
\SetKwFunction{Size}{size}
\SetKwFunction{Contains}{contains}
\SetKwFunction{Insert}{insert}
\SetKwFunction{Remove}{remove}
\SetKwFunction{Union}{union}
\SetKwFunction{Intersection}{intersection}
\SetKwFunction{Difference}{Difference}

\caption{\protect\Set}

\comment{Restituisce la cardinalit\`a dell'insieme}
\Int \Size{}\;
\comment{Restituisce True se $x$ \`e contenuto nell'insieme}
\Boolean \Contains{\Item $x$}\;
\comment{Inserisce $x$ nell'insieme se non \`e gi\`a presente}
\Insert{\Item $x$}\;
\comment{Rimuove $x$ dall'insieme se \`e presente}
\Remove{\Item $x$}\;
\comment{Restituisce un nuovo insieme che \`e l'unione di $A$ e $B$}
\Set \Union{\Set $A$, \Set $B$}\;
\comment{Restituisce un nuovo insieme che \`e l'intersezione di $A$ e $B$}
\Set \Intersection{\Set $A$, \Set $B$}\;
\comment{Restituisce un nuovo insieme che \`e la differenza tra $A$ e $B$}
\Set \Difference{\Set $A$, \Set $B$}\;
\end{algorithm}

\section{Dizionari}
Un dizionario \`e una struttura dati che rappresenta il concetto matematico di relazione univoca $R:D\rightarrow C$ o associazione chiave-valore. L'insieme
$D$ \`e il dominio ed \`e costituito dalle chiavi, l'insieme $C$ \`e il codominio ed \`e costituito dai valori.
\subsubsection{Operazioni ammesse}
\begin{itemize}
\item Ottenere il valore associato ad una particolare chiave se presente altrimenti \textbf{nil}.
\item Inserire una nuova associazione chiave-valore cancellando eventualmente associazioni precedenti per la stessa chiave.
\item Rimuovere un'associazione chiave-valore esistente.
\end{itemize}
\subsubsection{Specifica}
\begin{algorithm}
\DontPrintSemicolon
\SetKwData{Item}{Item}
\SetKwData{Dictionary}{Dictionary}
\SetKwComment{comment}{$\%$}{}
\SetKwFunction{Insert}{insert}
\SetKwFunction{Remove}{remove}
\SetKwFunction{Lookup}{lookup}

\caption{\protect\Dictionary}

\comment{Restituisce il valore associato alla chiave $k$ se presente, \textbf{nil} altrimenti}
\Item \Lookup{\Item $k$}\;
\comment{Associa il valore $v$ alla chiave $k$}
\Insert{\Item $k$, \Item $v$}\;
\comment{Rimuove l'associazione della chiave $k$}
\Remove{\Item $k$}\;
\end{algorithm}

\section{Alberi e grafi}
\subsection{Alberi ordinati}
Un albero ordinato \`e dato da un insieme finito di elementi detti nodi, uno dei quali \`e designato come radice e i rimanenti, se esistono, sono 
partizionati in insiemi ordinati e disgiunti, anch'essi alberi ordinati.
\subsection{Grafi}
La struttura di un grafo \`e composta da un insieme di elementi detti nodi o vertici e un insieme di coppie (ordinate o no) di nodi detti archi.
\subsection{Operazioni}
Tutte le operazioni su grafi e alberi ruotano intorno alla possibilit\`a di effettuare visite su di essi.
\section{Lista}
Una lista o linked list \`e una struttura dati contenente una sequenza di nodi contenente dati arbitrari, $1-2$ puntatori all'elemento successivo e/o a 
quello precedente. La contiguit\`a nella lista \`e diversa dalla contiguit\`a in memoria e tutte le operazioni su una lista hanno costo $O(1)$.
\subsubsection{Possibili implementazioni}
Bidirezionale o monodirezionale: in base al numero di puntatori e se puntano a quello successivo e/o a quello precedente.
Con sentinella o senza sentinella: in base alla presenza di una sentinella, una struttura che punta sempre al primo elemento della lista.
Circolare o non circolare: si dice circolare se il puntatore dell'ultimo elemento punta al primo invece di essere \textbf{nil}.
\subsubsection{Lista bidirezionale con sentinella}
\begin{algorithm}

\DontPrintSemicolon
\SetKwData{List}{List}
\SetKwData{Item}{Item}
\SetKwData{Pos}{Pos}
\SetKw{Boolean}{boolean}
\SetKwComment{comment}{$\%$}{}
\SetKwFunction{ListConst}{List}
\SetKwFunction{IsEmpty}{isEmpty}
\SetKwFunction{Head}{head}
\SetKwFunction{Tail}{tail}
\SetKwFunction{Next}{next}
\SetKwFunction{Prev}{prev}
\SetKwFunction{Finished}{finished}
\SetKwFunction{Read}{read}
\SetKwFunction{Write}{write}
\SetKwFunction{Insert}{insert}
\SetKwFunction{Remove}{remove}

\SetKwProg{Fn}{}{}{}

\SetKw{Delete}{delete}
\SetKw{New}{new}
\SetKw{Return}{return}
\SetKw{This}{this}
\SetKw{Space}{ }

\caption{\protect\List}

\begin{multicols}{2}
\List pred  \comment{Predecessore} 
\List succ  \comment{Successore} 
\Item value \comment{Valore}
\List\Space \Fn{\ListConst{}}{
	\List t=\New \List\;
	t.pred = t\;
	t.succ = t\;
	\Return t\;
}
\Boolean\Space\Fn{\IsEmpty{}}{
	\Return pred = succ = \This\;
}
\Pos\Space\Fn{\Head{}}{
	\Return succ\;
}
\Pos\Space\Fn{\Tail{}}{
	\Return pred\;
}
\Pos\Space\Fn{\Next{\Pos p}}{
	\Return p.succ\;
}
\Pos\Space\Fn{\Prev{\Pos p}}{
	\Return p.pred\;
}
\Boolean\Space\Fn{\Finished{\Pos p}}{
	\Return (p == \This)\;
}
\Item\Space\Fn{\Read{\Pos p}}{
	\Return p.value\;
}
\Item\Space\Fn{\Write{\Pos p, \Item v}}{
	p.value = v\;
}
\Pos\Space\Fn{\Insert{\Pos p, \Item v}}{
	\List t = \ListConst{}\;
	t.value = v\;
	t.pred = p.pred\;
	p.pred.succ = t\;
	t.succ = p\;
	p.pred = t\;
	\Return t\;
}
\Pos\Space\Fn{\Remove{\Pos p}}{
	p.pred.succ = p.succ\;
	p.succ.pred = p.pred\;
	\List t = p.succ\;
	\Delete p\;
	\Return t\; 
}
\end{multicols}
\end{algorithm}

\section{Pila}
Una pila o stack \`e una struttura dati lineare e dinamica in cui l'elemento rimosso dall'operazione di cancellazione \`e determinato, ovvero quello che 
per meno tempo \`e rimasto nell'insieme (LIFO: last-in, first-out). Possibili implementazioni sono una lista bidirezionale con un puntatore all'elemento 
top o tramite vettore, con dimensione limitata ma overhead pi\`u basso. 
\subsubsection{Specifica}
\begin{algorithm}[H]
\caption{Stack}
\DontPrintSemicolon
\SetKw{Boolean}{boolean}
\SetKw{True}{True}
\SetKwData{Item}{Item}
\SetKwComment{comment}{$\%$}{}
\SetKwFunction{IsEmpty}{isEmpty}
\SetKwFunction{Push}{push}
\SetKwFunction{Pop}{pop}
\SetKwFunction{Top}{top}

\comment{Restituisce \True se la pila \`e vuota}
\Boolean \IsEmpty{}\;
\comment{Inserisce $v$ in cima alla pila}
\Push{\Item $v$}\;
\comment{Estrae l'elemento in cima alla pila e lo restituisce}
\Item \Pop{}\;
\comment{Legge l'elemento in cima alla pila}
\Item \Top{}\;
\end{algorithm}

\subsubsection{Lista basata su vettore}
\begin{algorithm}[H]
\DontPrintSemicolon
\SetKwComment{comment}{$\%$}{}
\SetKwData{Item}{Item}
\SetKwData{Stack}{Stack}
\SetKw{Boolean}{boolean}
\SetKw{Int}{int}
\SetKw{Return}{return}
\SetKw{Prec}{precondition}
\SetKw{New}{new}
\SetKw{Space}{ }
\SetKwProg{Fn}{}{}{}
\SetKwFunction{StackCos}{Stack}
\SetKwFunction{IsEmpty}{isEmpty}
\SetKwFunction{Push}{push}
\SetKwFunction{Pop}{pop}
\SetKwFunction{Top}{top}

\caption{\protect\Stack}

\begin{multicols}{2}
\Item[] A   \comment{Elementi}
\Int n		\comment{Cursore}
\Int m		\comment{Dimensione massima}

\Stack\Space\Fn{\StackCos{\Int dim}}{
	\Stack t = \New \Stack\;
	t.A = \New \Int[$1\dots$dim]\;
	t.m = dim\;
	t.n = 0\;
	\Return t\;
}
\Item\Space\Fn{\Top{}}{
	\Prec : $n>0$\;
	\Return A[n]\;
}
\Boolean \Space\Fn{\IsEmpty{}}{
	\Return n == 0\;
}
\Item\Space\Fn{\Pop{}}{
	\Prec : $n>0$\;
	\Item t = A[n]\;
	n = n - 1\;
	\Return t\;
}
\Fn{\Push{\Item v}}{
	\Prec : $n<m$\;
	n = n + 1\;
	A[n] = v\;
}
\end{multicols}
\end{algorithm}

\section{Coda}
Una coda \`e una struttura dati lineare e dinamica in cui l'elemento rimosso dall'operazione di cancellazione \`e determinato, ovvero quello che per pi\`u 
tempo \`e rimasto nell'insieme (FIFO: first-in, first-out, questo tipo di politica si dice fair). Si pu\`o implementare attraverso liste monodirezionali 
con il puntatore head per l'estrazione e il puntatore tail per l'inserimento o tramite array circolari, con dimensione limitata ma overhead pi\`u basso.
Per i secondi la circolarit\`a \`e ottenuta attraverso l'operazione modulo e si deve prestare attenzione ai problemi di overflow.
\subsubsection{Specifica}
\begin{algorithm}[H]
\DontPrintSemicolon
\SetKw{Boolean}{boolean}
\SetKw{True}{True}
\SetKwData{Item}{Item}
\SetKwData{Queue}{Queue}
\SetKwComment{comment}{$\%$}{}
\SetKwFunction{IsEmpty}{isEmpty}
\SetKwFunction{Enqueue}{enqueue}
\SetKwFunction{Dequeue}{dequeue}
\SetKwFunction{Top}{top}

\caption{\protect\Queue}

\comment{Restituisce \True se la coda \`e vuota}
\Boolean \IsEmpty{}\;
\comment{Inserisce $v$ in fondo alla coda}
\Enqueue{\Item $v$}\;
\comment{Estrae l'elemento in testa alla coda e lo restituisce}
\Item \Dequeue{}\;
\comment{Legge l'elemento in testa alla coda}
\Item \Top{}\;
\end{algorithm}

\subsubsection{Coda basata su vettore circolare}
\begin{algorithm}[H]
\DontPrintSemicolon
\SetKwComment{comment}{$\%$}{}
\SetKwData{Item}{Item}
\SetKwData{Queue}{Queue}
\SetKw{Boolean}{boolean}
\SetKw{Int}{int}
\SetKw{Return}{return}
\SetKw{Prec}{precondition}
\SetKw{New}{new}
\SetKw{Space}{ }
\SetKwProg{Fn}{}{}{}
\SetKwFunction{QueueCos}{Queue}
\SetKwFunction{IsEmpty}{isEmpty}
\SetKwFunction{Dequeue}{dequeue}
\SetKwFunction{Enqueue}{enqueue}
\SetKwFunction{Top}{top}

\caption{\protect\Queue}

\begin{multicols}{2}
\Item[] A   \comment{Elementi}
\Int n		\comment{Dimensione attuale}
\Int testa	\comment{Testa}
\Int m		\comment{Dimensione massima}
\Queue\Space\Fn{\QueueCos{\Int dim}}{
	\Queue t = \New \Queue\;
	t.A = \New \Int[$1\dots$dim - 1]\;
	t.m = dim\;	
	t.testa = 0\;
	t.n = 0\;
	\Return t\;
}
\Item\Space\Fn{\Top{}}{
	\Prec : $n>0$\;
	\Return A[testa]\;
}
\Boolean \Space\Fn{\IsEmpty{}}{
	\Return n == 0\;
}
\Item\Space\Fn{\Dequeue{}}{
	\Prec : $n>0$\;
	\Item t = A[testa]\;
	testa = (testa + 1) $\mod$ m\;
	n = n - 1\;
	\Return t\;
}
\Fn{\Enqueue{\Item v}}{
	\Prec : $n<m$\;
	A[(testa + n) $\mod$ m] = v\;	
	n = n + 1\;
	
}
\end{multicols}
\end{algorithm}

