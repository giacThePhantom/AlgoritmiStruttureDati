\chapter{Divide-et-impera}
In questo metodo di risoluzione problemi ci sono tre fasi: 
\begin{itemize}
\item Divide: si divide il problema in sotto-problemi pi\`u piccoli e indipendenti.
\item Impera: si risolvono i sotto-problemi ricorsivamente.
\item Combina: si uniscono le soluzioni dei sottoproblemi. 
\end{itemize}
\section{Le torri di Hanoi}
Le torri di Hanoi \`e un gioco matematico costituito da tre pioli con $n$ dischi di dimensioni diverse. Inizialmente i dischi sono impilati in ordine decrescente nel piolo di sinistra. Lo scopo del gioco \`e impilare
in ordine decrescente i dischi sul piolo di destra; senza mai impilare un disco pi\`u grande su uno pi\`u piccolo, muovendo al massimo un disco alla volta e utilizzando il piolo centrale come appoggio. 
\begin{multicols}{2}
\begin{algorithm}[H]
\DontPrintSemicolon
\SetKwComment{comment}{$\%$}{}

\SetKw{Int}{int}
\SetKw{Boolean}{boolean}
\SetKw{New}{new}
\SetKw{True}{true}
\SetKw{False}{false}
\SetKw{Not}{not}
\SetKw{And}{and}
\SetKw{Down}{down}
\SetKw{To}{to}
\SetKw{New}{new}
\SetKw{Return}{return}
\SetKw{Print}{print}

\SetKwData{Item}{Item}
\SetKwData{Mfset}{Mfset}
\SetKwData{Graph}{Graph}
\SetKwData{N}{n}
\SetKwData{Space}{ }
\SetKwData{Parent}{parent}
\SetKwData{Rank}{rank}

\SetKwFunction{Hanoi}{hanoi}
\SetKwFunction{}{}
\SetKwFunction{}{}
\SetKwFunction{}{}
\SetKwFunction{}{}
\SetKwFunction{}{}

\caption{\protect\Hanoi{\protect\Int n, \protect\Int src, \protect\Int middle}}
\uIf{n = 1}{
	\Print srt $\rightarrow$ dest\;	
}
\Else{
	\Hanoi{n -1, src, middle, dest}\;
	\Print src $\rightarrow$ dest\;
	\Hanoi{n -1, middle, dest, src}\;
}


\end{algorithm}

Questo algoritmo trova la soluzione ottima, la ricorrenza \`e $$T(n) = 2T(n-1)+1$$ con costo computazionale $O(2^n)$.
\end{multicols}
\section{Quicksort}
Quicksort \`e un algoritmo di ordinamento basato su divide-et-impera, nel caso medio \`e $O(n\log n)$, nel caso pessimo $O(n^2)$. Il fattore costante \`e migliore di quello di Mergesort, non utilizza memoria
addizionale e presenta tecniche euristiche per evitare il caso pessimo.
\subsection{Caratterizzazione}
\subsubsection{Input}
Un vettore $A[1\dots n]$, indici \emph{start}, \emph{end} tali che $1\le start\le end\le n$.
\subsubsection{Divide}
Si sceglie un valore $p\in A[start\dots end]$ detto perno o pivot. Si spostano gli elementi del vettore $A[start\dots end]$ in modo tale che:
\begin{itemize}
\item $\forall i\in [start\dots j-1]: A[i]\le p$.
\item $\forall i\in [j+1\dots end]: A[i]\ge p$.
\end{itemize}
L'indice $j$ viene calcolato in modo tale da rispettare tale condizione, il perno viene messo in posizione $A[j]$. 
\subsubsection{Impera}
Ordina i due sottovettori $A[start\dots j-1]$ e $A[j+1\dots end]$ richiamando ricorsivamente Quicksort.
\subsubsection{Combina}
Non fa nulla: il primo vettore, $A[j]$ e il secondo vettore formano gi\`a un vettore ordinato. 
\begin{multicols}{2}
\begin{algorithm}[H]
\DontPrintSemicolon
\SetKwComment{comment}{$\%$}{}

\SetKw{Int}{int}
\SetKw{Boolean}{boolean}
\SetKw{New}{new}
\SetKw{True}{true}
\SetKw{False}{false}
\SetKw{Not}{not}
\SetKw{And}{and}
\SetKw{Down}{down}
\SetKw{To}{to}
\SetKw{New}{new}
\SetKw{Return}{return}
\SetKw{Print}{print}

\SetKwData{Item}{Item}
\SetKwData{Mfset}{Mfset}
\SetKwData{Graph}{Graph}
\SetKwData{N}{n}
\SetKwData{Space}{ }
\SetKwData{Parent}{parent}
\SetKwData{Rank}{rank}

\SetKwFunction{Hanoi}{hanoi}
\SetKwFunction{Pivot}{pivot}
\SetKwFunction{}{}
\SetKwFunction{}{}
\SetKwFunction{}{}
\SetKwFunction{}{}

\caption{\protect\Int \protect\Pivot{\protect\Item[] A, \protect\Int start, \protect\Int end}}
\Item p = A[start]\;
\Int j = start\;
\For{\Int i = start +1 \To end}{
	\If{A[i] $<$ p}{
		j += 1\;
		A[i] $\leftrightarrow$ A[j]\;	
	}
}
A[start] = A[j]\;
A[j] = p\;
\Return j\;
\end{algorithm}

\columnbreak
\begin{algorithm}[H]
\DontPrintSemicolon
\SetKwComment{comment}{$\%$}{}

\SetKw{Int}{int}
\SetKw{Boolean}{boolean}
\SetKw{New}{new}
\SetKw{True}{true}
\SetKw{False}{false}
\SetKw{Not}{not}
\SetKw{And}{and}
\SetKw{Down}{down}
\SetKw{To}{to}
\SetKw{New}{new}
\SetKw{Return}{return}
\SetKw{Print}{print}

\SetKwData{Item}{Item}
\SetKwData{Mfset}{Mfset}
\SetKwData{Graph}{Graph}
\SetKwData{N}{n}
\SetKwData{Space}{ }
\SetKwData{Parent}{parent}
\SetKwData{Rank}{rank}

\SetKwFunction{QuickSort}{Quicksort}
\SetKwFunction{Pivot}{pivot}
\SetKwFunction{}{}
\SetKwFunction{}{}
\SetKwFunction{}{}
\SetKwFunction{}{}

\caption{\protect\QuickSort{\protect\Item[] A, \protect\Int start, \protect\Int end}}
\If{start $<$ end}{
	\Int j = \Pivot{A, start, end}\;
	\QuickSort{A, start, j -1 }\;
	\QuickSort{A, j+1, end}\;
}
\end{algorithm}

\end{multicols}
\subsection{Complessit\`a computazionale}
Il costo di \emph{pivot()} \`e $\Theta(n)$, mentre il costo di \emph{Quicksort()} dipende dal partizionamento:
\begin{itemize}
\item Partizionamento peggiore: dato un vettore di dimensione $n$ viene diviso i due sotto-problemi di dimensione $0$ e $n-1$. $$T(n) = T(n-1) + T(0) + 
\Theta(n) = \Theta(n^2)$$
\item Partizionamento migliore: dato un vettore di dimensione $n$ viene sempre diviso in due sotto-problemi di dimensione $\frac{n}{2}$.  $$T(n) = 
2T(\frac{n}{2}) + \Theta(n) = \Theta(n\log n)$$
\end{itemize}
Nel caso medio si alternano partizionamenti peggiori e migliori con il caso medio nel numero maggiore, pertanto nel caso medio ha complessit\`a $O(n\log n)
$.
\section{Conclusioni}
Si deve applicare divide-et-impera quando i passi "divide" e "combina" sono semplici e i costi sono migliori del corrispondente algoritmo iterativo. Presenta inoltre una pi\`u facile parallelizzazione e un utilizzo
ottimale della cache ("cache oblivious"). 