\chapter{Insiemi e dizionari}
\section{Insiemi}
\subsection{Insiemi realizzati con vettori booleani}
\subsubsection{Implementazione}
\begin{algorithm}[H]
\DontPrintSemicolon
\SetKwComment{comment}{$\%$}{}
\SetKw{Int}{int}
\SetKw{Boolean}{boolean}
\SetKw{Nil}{nil}
\SetKw{New}{new}
\SetKw{Return}{return}
\SetKw{Deleted}{deleted}
\SetKw{True}{true}
\SetKw{False}{false}
\SetKw{To}{to}
\SetKw{And}{and}
\SetKw{Not}{not}
\SetKw{Or}{or}
\SetKwData{Item}{Item}
\SetKwData{Set}{Set}
\SetKwData{Space}{ }
\SetKwProg{Fn}{}{}{}
\SetKwFunction{Max}{max}
\SetKwFunction{Min}{min}
\SetKwFunction{SetCos}{Set}
\SetKwFunction{Contains}{contains}
\SetKwFunction{Size}{size}
\SetKwFunction{Insert}{insert}
\SetKwFunction{Remove}{remove}
\SetKwFunction{Union}{union}
\SetKwFunction{Intersection}{intersection}
\SetKwFunction{Difference}{difference}

\caption{\protect\Set (vettore booleano)}
\begin{multicols}{2}
\Boolean[] V\;
\Int size\;
\Int dim\;

\Set\Space\Fn{\SetCos{\Int m}}{
	\Set t = \New \Set\;
	t.size = 0\;
	t.dim = =m\;
	t.V = [\False] * m\;
	\Return t\;
}

\Boolean\Space\Fn{\Contains{\Int x}}{
	\uIf{1 $\le$ x $\le$ dim}{
		\Return V[x]\;	
	}
	\Else{
		\Return \False\;	
	}
}

\Int\Space\Fn{\Size{}}{
	\Return size\;
}

\Fn{\Insert{\Int x}}{
	\If{1 $\le$ x $\le$ dim}{
		\If{\Not V[x]}{
			size += 1\;
			V[x] = \True\;		
		}	
	}
}

\Fn{\Remove{\Int x}}{
	\If{1 $\le$ x $\le$ dim}{
		\If{V[x]}{
			size == 1\;
			V[x] = \False\;		
		}	
	}
}

\Set\Space\Fn{\Union{\Set A, \Set B}}{
	\Set C= \SetCos{\Max{A.dim, B.dim}}\;
	\For{i = 1 \To A.dim}{
		\If{A.\Contains{i}}{
			C.\Insert{i}\;		
		}	
	}
	\For{i = 1 \To B.dim}{
		\If{B.\Contains{i}}{
			C.\Insert{i}\;		
		}	
	}
	\Return C\;
}

\Set\Space\Fn{\Intersection{\Set A, \Set B}}{
	\Set C= \SetCos{\Min{A.dim, B.dim}}\;
	\For{i = 1 \To \Min{A.dim, B.dim}}{
		\If{A.\Contains{i} \And B.\Contains{i}}{
			C.\Insert{i}\;		
		}	
	}
	\Return C\;
}

\Set\Space\Fn{\Difference{\Set A, \Set B}}{
	\Set C= \SetCos{A.dim}\;
	\For{i = 1 \To A.dim}{
		\If{A.\Contains{i} \And \Not B.\Contains{i}}{
			C.\Insert{i}\;		
		}	
	}
	\Return C\;
}
\end{multicols}
\end{algorithm}

\subsubsection{Caratteristiche}
Gli insiemi possono essere memorizzati come un vettore di $m$ elementi, se si vogliono salvare gli interi da $1$ a $m$. Questo tipo di implementazione \`e
molto semplice ed \`e efficiente verificare se un elemento appartiene ad un insieme. Come svantaggi presenta lo spreco di memoria, in quanto la memoria
occupata \`e sempre $O(m)$ indipendentemente dagli elementi salvati. E alcune operazioni sono inefficienti in $O(m)$. 
\subsection{Insiemi realizzati con liste}
\subsubsection{Liste non ordinate}
Operazioni di ricerca, inserimento e cancellazione $O(n)$, operazioni di inserimento assumendo assenza $O(1)$. Operazioni di unione intersezione, differenza
$O(nm)$.
\begin{algorithm}
\DontPrintSemicolon
\SetKwComment{comment}{$\%$}{}
\SetKw{Int}{int}
\SetKw{Boolean}{boolean}
\SetKw{Nil}{nil}
\SetKw{New}{new}
\SetKw{Return}{return}
\SetKw{Deleted}{deleted}
\SetKw{True}{true}
\SetKw{False}{false}
\SetKw{To}{to}
\SetKw{And}{and}
\SetKw{Not}{not}
\SetKw{Or}{or}
\SetKwData{Item}{Item}
\SetKwData{Set}{Set}
\SetKwProg{Fn}{}{}{}
\SetKwFunction{Max}{max}
\SetKwFunction{Min}{min}
\SetKwFunction{SetCos}{Set}
\SetKwFunction{Contains}{contains}
\SetKwFunction{Size}{size}
\SetKwFunction{Insert}{insert}
\SetKwFunction{Remove}{remove}
\SetKwFunction{Union}{union}
\SetKwFunction{Intersection}{intersection}
\SetKwFunction{Difference}{difference}

\caption{\protect\Set \protect\Difference{\protect\Set A, \protect\Set B}}

\Set C = \Set{}\;
\ForEach{s $\in$ A}{
	\If{\Not B.\Contains{s}}{
		C.\Insert{s}\;	
	}
	\Return C\;
}
\end{algorithm}

\subsubsection{Liste ordinate}
Ricerca $O(n)$ per liste e $O(\log n)$ per i vettori, inserimento e cancellazione $O(n)$, unione, intersezione e differenza $O(n)$.\\
\begin{algorithm}[H]
\DontPrintSemicolon
\SetKwComment{comment}{$\%$}{}
\SetKw{Int}{int}
\SetKw{Boolean}{boolean}
\SetKw{Nil}{nil}
\SetKw{New}{new}
\SetKw{Return}{return}
\SetKw{Deleted}{deleted}
\SetKw{True}{true}
\SetKw{False}{false}
\SetKw{To}{to}
\SetKw{And}{and}
\SetKw{Not}{not}
\SetKw{Or}{or}
\SetKwData{Item}{Item}
\SetKwData{Set}{List}
\SetKwData{Pos}{Pos}
\SetKwProg{Fn}{}{}{}
\SetKwFunction{Max}{max}
\SetKwFunction{Min}{min}
\SetKwFunction{SetCos}{Set}
\SetKwFunction{Contains}{contains}
\SetKwFunction{Size}{size}
\SetKwFunction{Insert}{insert}
\SetKwFunction{Remove}{remove}
\SetKwFunction{Union}{union}
\SetKwFunction{Intersection}{intersection}
\SetKwFunction{Difference}{difference}
\SetKwFunction{Head}{head}
\SetKwFunction{Next}{next}
\SetKwFunction{Read}{read}
\SetKwFunction{Tail}{tail}
\SetKwFunction{Finished}{finished}
\caption{\protect\Set \protect\Intersection{\protect\Set A, \protect\Set B}}

\Set C = \Set{}\;
\Pos p = A.\Head{}\;
\Pos q = B.\Head{}\;

\While{\Not A.\Finished{p} \And \Not B.\Finished{q}}{
	\uIf{A.\Read{p} = B.\Read{q}}{
		C.\Insert{C.\Tail{}, A.\Read{p}}\;
		p = A.\Next{p}\;
		q = B.\Next{q}\;	
	}
	\uElseIf{A.\Read{p} < B.\Read{q}}{
		p = A.\Next{p}\;
	}
	\Else{
		q = B.\Next{q}\;	
	}
}
\Return C\;
\end{algorithm}

\subsection{Strutture dati complesse}
\subsubsection{Alberi bilanciati}
Si ottengono insiemi con ordinamento, con ricerca, inserimento, cancellazione $O(\log n)$, iterazione $O(n)$.
\subsubsection{Tabelle hash}
Si ottengono insiemi senza ordinamento, con ricerca, inserimento, cancellazione $O(1)$, iterazione $O(m)$. 
\section{Bloom filters}
I bloom filters sono una via di mezzo tra gli insiemi realizzati con i vettori booleani e le tabelle hash, sono una struttura dati dinamica con bassa
occupazione di memoria che non offre la possibilit\`a di cancellazioni, nessuna memorizzazione e d\`a risposte probabilistiche. 
\subsection{Specifica}
L'operazione di \emph{insert(x)} inserisce l'elemento $x$ nel bloom filter. \emph{\textbf{booleans} contains(x)} se restituisce \textbf{false} l'elemento 
non sicuramente \`e presente, ma \`e possibile che ci siano dei falsi positivi. Si deve fare un trade-off tra occupazione di memoria e probabilit\`a di un
falso positivo. Indicata con $\varepsilon$ la probabilit\`a di un falso positivo, i bloom filters richiedono $1.44\log_2(\frac{1}{\varepsilon})$ bit per 
elemento inserito. 
\subsection{Applicazioni}
In chrome vengono utilizzati per indicare i siti con possibile malware, o in generale vengono utilizzati quando una verifica locale permette di evitare 
operazioni di I/O pi\`u costose. I falsi positivi inoltre possono essere utilizzati per mascherare un messaggio e garantire cos\`i un certo livello di 
privacy.
\newpage
\subsection{Implementazione}
Vengono implementati attraverso un vettore booleando $A$ di $m$ bit inizializzato a \textbf{false} e $k$ funzioni di hash $H_1,\dots, H_k:U\rightarrow [0, 
\dots, m-1]$. 
\begin{multicols}{2}
\begin{algorithm}[H]
\DontPrintSemicolon
\SetKwComment{comment}{$\%$}{}
\SetKw{Int}{int}
\SetKw{Boolean}{boolean}
\SetKw{Nil}{nil}
\SetKw{New}{new}
\SetKw{Return}{return}
\SetKw{Deleted}{deleted}
\SetKw{True}{true}
\SetKw{False}{false}
\SetKw{To}{to}
\SetKw{And}{and}
\SetKw{Not}{not}
\SetKw{Or}{or}
\SetKwData{Item}{Item}
\SetKwFunction{Insert}{insert}
\SetKwFunction{Hash}{$H_i$}
\caption{\protect\Insert{x}}
\For{i = 1 \To k}{
	A[\Hash{x}] = \True\;
}
\end{algorithm}

\columnbreak
\begin{algorithm}[H]
\DontPrintSemicolon
\SetKwComment{comment}{$\%$}{}
\SetKw{Int}{int}
\SetKw{Boolean}{boolean}
\SetKw{Nil}{nil}
\SetKw{New}{new}
\SetKw{Return}{return}
\SetKw{Deleted}{deleted}
\SetKw{True}{true}
\SetKw{False}{false}
\SetKw{To}{to}
\SetKw{And}{and}
\SetKw{Not}{not}
\SetKw{Or}{or}
\SetKwData{Item}{Item}
\SetKwFunction{Insert}{contains}
\SetKwFunction{Hash}{$H_i$}
\caption{\protect\Boolean \protect\Insert{x}}
\For{i = 1 \To k}{
	\If{A[\Hash{x}] == \False}{
		\Return \False\;
	}
}
\Return \True\;
\end{algorithm}

\end{multicols}
\subsection{Caratterizzazione matematica}
Dati $n$ oggetti, $m$ bit e $k$ funzioni hash, la probabilit\`a di un falso positivo \`e:
\begin{equation*}
\varepsilon=(1-e^{-k\frac{n}{m}})^k
\end{equation*}
Dati $n$ oggetti e $m$ bit il valore ottimale per $k$ \`e:
\begin{equation*}
k=\dfrac{m}{n}\ln 2
\end{equation*}
Dati $n$ oggetti e una probabilit\`a di falsi positivi $\varepsilon$ il numero di bit $m$ richiesti \`e pari a:
\begin{equation*}
m=\dfrac{n\ln\varepsilon}{(\ln 2)^2}
\end{equation*}