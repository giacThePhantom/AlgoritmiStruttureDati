\chapter{Tecniche risolutive per problemi intrattabili}
Per la risoluzione di alcuni problemi si rende necessario rinunciare a delle caratteristiche:
\begin{itemize}
	\item Generalit\`a: algoritmi pseudo-polinomiali che funzionano solo per alcuni casi particolari dell'input.
	\item Ottimalit\`a: algoritmi di approssimazione che garantiscono di ottenere soluzioni vicine alla soluzione ottimale.
	\item Efficienza: algoritmi esponenziali branch-\&-bound che limitano lo spazio di ricerca con un'accurata potatura.
	\item Formalit\`a: algoritmi euristici di solito basati su tecniche greedy o di ricerca locale che forniscano sperimentalmente risultati buoni.
\end{itemize}
\section{Subset-Sum}
\paragraph{Problema}
Dati un vettore $A$ contenente $n$ interi positivi e un intero positivo $k$, verificare se esiste un sottoinsieme $S\subseteq\{1, \dots, n\}$ tale che $\sum\limits_{i\in S}a[i] = k$.
\paragraph{Soluzioni}
Utilizzando backtracking si \`e risolta la versione di ricerca di questo problema mentre questa \`e la versione decisionale, su cui ci si concentrer\`a. 
\subsection{Programmazione dinamica}
Si definisce una tabella booleana $DP[0\dots n][0\dots k]$ tale per cui $DP[i][r]$ \`e uguale a \emph{\textbf{true}} se esiste un sottoinsieme dei primi $i$ valori memorizzati in $A$
la cui somma \`e pari a $r$, \emph{\textbf{false}} altrimenti.
$$DP[i][r] = 
\begin{cases}
	true \quad\quad\quad & r = 0\\
	false & r > o \land i = 0\\
	DP[i - 1][r] & r > 0 \land i > 0 \land A[i] > r\\
	DP[i - 1][r] \lor DP[i - 1][r - A[i] & r > 0 \land i > - \land A[i]\le r\\
\end{cases}$$
\begin{algorithm}[H]
\DontPrintSemicolon
\SetKwComment{comment}{$\%$}{}
\SetKwRepeat{Do}{do}{while}
\SetKw{Int}{int}
\SetKw{Float}{float}
\SetKw{Boolean}{boolean}
\SetKw{New}{new}
\SetKw{True}{true}
\SetKw{False}{false}
\SetKw{Not}{not}
\SetKw{And}{and}
\SetKw{Or}{or}
\SetKw{Down}{down}
\SetKw{To}{to}
\SetKw{New}{new}
\SetKw{Return}{return}
\SetKw{Nil}{nil}
\SetKw{Print}{print}
\SetKw{Void}{void}

\SetKwData{Item}{Item}
\SetKwData{Tree}{Tree}
\SetKwData{PriorityQueue}{PriorityQueue}
\SetKwData{Mfset}{Mfset}
\SetKwData{Graph}{Graph}
\SetKwData{N}{n}
\SetKwData{Space}{ }
\SetKwData{Parent}{parent}
\SetKwData{Rank}{rank}
\SetKwData{Set}{Set}
\SetKwData{List}{List}
\SetKwData{F}{f}
\SetKwData{Left}{left}
\SetKwData{Right}{right}
\SetKwData{Mfset}{Mfset}
\SetKwData{PriorityItem}{PriorityItem}
\SetKwData{Node}{Node}
\SetKwData{Point}{Point}
\SetKwData{Stack}{Stack}

\SetKwFunction{Max}{max}
\SetKwFunction{Min}{min}
\SetKwFunction{Insert}{insert}
\SetKwFunction{ListCos}{List}
\SetKwFunction{Head}{head}
\SetKwFunction{Len}{len}
\SetKwFunction{SetCos}{Set}
\SetKwFunction{TreeCos}{Tree}
\SetKwFunction{MinPriorityQueue}{MinPriorityQueue}
\SetKwFunction{DeleteMin}{deleteMin}
\SetKwFunction{MfsetCos}{Mfset}
\SetKwFunction{Find}{find}
\SetKwFunction{Merge}{merge}
\SetKwFunction{IsEmpty}{isEmpty}
\SetKwFunction{Adj}{adj}
\SetKwFunction{Push}{push}
\SetKwFunction{Pop}{pop}
\SetKwFunction{Top}{top}
\SetKwFunction{Decrease}{decrease}
\SetKwFunction{StackCos}{StackCos}


\SetKwProg{Fn}{}{}

\SetKwFunction{SubsetSum}{subsetSum}
\SetKwFunction{}{}
\SetKwFunction{}{}
\SetKwFunction{}{}
\SetKwFunction{}{}
\SetKwFunction{}{}

\caption{\protect\Boolean \protect\SubsetSum{\protect\Int[] A, \protect\Int n, \protect\Int k}}
\Boolean[][] DP = \New \Boolean$[0\dots n][0\dots k] = \{$\False$\}$\;
\For{\Int i = 0 \To n}{
	DP[i][0] = \True\comment{r = 0}
}
\For{\Int r = 1 \To k}{
	DP[0][k] =\False\comment{r $>$ 0 $\land$ i = 0}
}
\For{\Int i = 1 \To n}{
	\For{\Int r = 1 \To A[i] - 1}{
		DP[i][r] =DP[i - 1][r]\comment{A[i] $>$ r}
	}
	\For{\Int r = A[i] \To k}{
		DP[i][r] = DP[i][r] = DP[i - 1][r] \Or DP[i - 1][r - A[i]]\comment{A[i] $\le$ r}
	}
}
\Return DP[n][k]\;
\end{algorithm}

\subsection{Backtracking}
\begin{algorithm}[H]
\DontPrintSemicolon
\SetKwComment{comment}{$\%$}{}
\SetKwRepeat{Do}{do}{while}
\SetKw{Int}{int}
\SetKw{Float}{float}
\SetKw{Boolean}{boolean}
\SetKw{New}{new}
\SetKw{True}{true}
\SetKw{False}{false}
\SetKw{Not}{not}
\SetKw{And}{and}
\SetKw{Or}{or}
\SetKw{Down}{down}
\SetKw{To}{to}
\SetKw{New}{new}
\SetKw{Return}{return}
\SetKw{Nil}{nil}
\SetKw{Print}{print}
\SetKw{Void}{void}

\SetKwData{Item}{Item}
\SetKwData{Tree}{Tree}
\SetKwData{PriorityQueue}{PriorityQueue}
\SetKwData{Mfset}{Mfset}
\SetKwData{Graph}{Graph}
\SetKwData{N}{n}
\SetKwData{Space}{ }
\SetKwData{Parent}{parent}
\SetKwData{Rank}{rank}
\SetKwData{Set}{Set}
\SetKwData{List}{List}
\SetKwData{F}{f}
\SetKwData{Left}{left}
\SetKwData{Right}{right}
\SetKwData{Mfset}{Mfset}
\SetKwData{PriorityItem}{PriorityItem}
\SetKwData{Node}{Node}
\SetKwData{Point}{Point}
\SetKwData{Stack}{Stack}

\SetKwFunction{Max}{max}
\SetKwFunction{Min}{min}
\SetKwFunction{Insert}{insert}
\SetKwFunction{ListCos}{List}
\SetKwFunction{Head}{head}
\SetKwFunction{Len}{len}
\SetKwFunction{SetCos}{Set}
\SetKwFunction{TreeCos}{Tree}
\SetKwFunction{MinPriorityQueue}{MinPriorityQueue}
\SetKwFunction{DeleteMin}{deleteMin}
\SetKwFunction{MfsetCos}{Mfset}
\SetKwFunction{Find}{find}
\SetKwFunction{Merge}{merge}
\SetKwFunction{IsEmpty}{isEmpty}
\SetKwFunction{Adj}{adj}
\SetKwFunction{Push}{push}
\SetKwFunction{Pop}{pop}
\SetKwFunction{Top}{top}
\SetKwFunction{Decrease}{decrease}
\SetKwFunction{StackCos}{StackCos}


\SetKwProg{Fn}{}{}

\SetKwFunction{SsRec}{ssRec}
\SetKwFunction{}{}
\SetKwFunction{}{}
\SetKwFunction{}{}
\SetKwFunction{}{}
\SetKwFunction{}{}

\caption{\protect\Boolean \protect\SsRec{\protect[] A, \protect\Int i, \protect\Int r}}
\uIf{r == 0}{
	\Return \True\;
}
\uElseIf{i == 0}{
	\Return \False\;
}
\uElseIf{A[i] $>$ r}{
	\Return \SsRec{A, i - 1, r}\;
}
\Else{
	\Return \SsRec{A, i - 1, r} \Or \SsRec{A, i - 1, r - A[i]}\;
}
\end{algorithm}

\subsection{Memoization}
\input{Pseudocodice/145_subSumMem}
\subsection{Complessit\`a}
\begin{itemize}
	\item Programmazione dinamica: $\Theta(nk)$.
	\item Backtracking: $O(2^n)$.
	\item Memoization con dizionario $O(nk), O(2^n)$.
\end{itemize}
$O(nk)$ non \`e una complessit\`a polinomiale in quanto $k$ \`e parte dell'input, non una sua dimensione. Questo viene rappresentato con $t = \lceil\log k\rceil$ cifre binarie, pertanto
$O(nk) = O(n\cdot 2^t)$, che \`e esponenziale.
\section{Problemi fortemente, debolmente $\mathbf{\mathbb{NP}}$-completi}
\subsubsection{Dimensioni del problema}
Dato un problema decisionale $R$ e una sua istanza $I$, si definisce la dimensione $d$ di $I$ \`e la lunghezza della stringa che codifica $I$, il valore $\#$ \`e il pi\`u grande numero 
intero che appare in $I$.
\subsection{Problemi}
\subsubsection{Cricca (CLIQUE)}
Dati un grafo non orientato e un intero $k$ si verifiche se esiste un sottoinsieme di almeno $k$ nodi tutti mutualmente adiacenti.
\subsubsection{Commesso viaggiatore (TSP)}
Date $n$ citt\`a e una matrice simmetrica $d$ di distanze positive, dove $d[i][j]$ \`e la distanza tra $i$ e $j$, trovare un provare un percorso che, partendo da una qualsiasi citt\`a
attraverso ogni citt\`a esattamente una volta e ritorni alla citt\`a di partenza in modo che la distanza totale percorsa sia minima.
\subsubsection{Esempi}
\begin{tabular}{|c|c|c|c|}
	\hline
	\textbf{Nome} & $\mathbf{I}$ & $\mathbf{\#}$ & $\mathbf{d}$ \\
	\hline
	SUBSET-SUM & $\{n, k, A\}$ & $\max\{n, k, \max(A)\}$ & $O(n\log\#)$\\
	\hline
	CLIQUE & $\{n, m, k, G\}$ & $\max\{n, m, k\}$ & $O(n + m + \log\#)$\\
	\hline
	TSP & $\{n, k, d\}$ & $\max\{n, k, \max(d)\}$ & $O(n^2\log\#)$ \\
	\hline
\end{tabular}
\subsection{Problema fortemente $\mathbf{\mathbb{NP}}$-completo}
Sia $R_{pol}$ il problema $R$ ristretto a quei dati d'ingresso per i quali $\#$ \`e limitato superiormente da $p(d)$ con $p$ una funzione polinomiale in $d$. $R$ \`e fortemente 
$\mathbb{NP}$-completo se $R_{pol}$ \`e $\mathbb{NP}$-completo.
\subsubsection{Cricca (CLIQUE)}
Si mostri come questo problema \`e fortemente $\mathbb{NP}$-completo: si prenda $k\le n$ (altrimenti la risposta \`e \textbf{false}), $\# = \max\{n, m, k\} = \max\{n, m\}$, 
$d = O(n+m+\log\#)=O(n+m)$, pertanto $\# = \max\{n, m\}$ \`e limitato superiormente da $O(n+m)$ quindi il problema ristretto \`e identico a CLIQUE che \`e $\mathbb{NP}$-completo.
\subsubsection{Commesso viaggiatore (TSP)}
Si dimostri che TSP \`e fortemente $\mathbb{NP}$-completo. Per assurdo si supponga che TSP sia debolmente $\mathbb{NP}$-completo, allora esiste una soluzione pseudo-polinomiale  e la 
si usa per risolvere un problema $\mathbb{NP}$-completo in tempo polinomiale, assurdo a meno che $\mathbb{P}=\mathbb{NP}$. 
\subsubsection{Circuito hamiltoniano (HAMILTONIAN-CIRCUIT)}
Si deve verificare se dato un grafo non orientato $G$ esista un circuito che attraversi ogni nodo una e una sola volta. 
\paragraph{Complessit\`a}
HAMILTONIAN-CIRCUIT \`e $\mathbb{NP}$-completo e uno dei $21$ problemi elencati nell'articolo di Karp.
\subsubsection{Dimostrazione che TSP \`e fortemente $\mathbf{\mathbb{NP}}$-completo}
Sia $G=(V, E)$ un grafo non orientato e si definisca una matrice di distanze a partire da $G$:
$$d[i][j] = 
\begin{cases}
	1 & (i, j)\in E\\
	2 & (i, j)\not\in E\\
\end{cases}$$
Il grafo $G$ ha un circuito hamiltoniano se e solo se \`e possibile trovare un percorso da commesso viaggiatore di costo $n$, se esistesse un algoritmo pseudopolinomiale $A$ per TSP, 
HAMILTONIAN-CIRCUIT potrebbe essere risolto da $A$ in tempo polinomiale, che \`e un assurdo.
\subsection{Problema debolmente $\mathbf{\mathbb{NP}}$-completo}
Se un problema $\mathbb{NP}$-completo non \`e fortemente $\mathbb{NP}$-completo allora \`e debolmente $\mathbb{NP}$-completo.
\subsubsection{Somma di sottoinsieme (SUBSET-SUM)}
Dati un vettore $A$ contenente $n$ interi positivi e un intero positivo $k$, verificare se esiste un sottoinsieme $S\subseteq\{1\dots n\}$ tale che $\sum\limits_{i\in S}a[i] = k$.
\paragraph{Si verifiche che SUBSET-SUM \`e debolmente $\mathbf{\mathbb{NP}}$-completo}
Si consideri $\forall A[i]\le k$ con i valori maggiori di $k$ esclusi, se $k=O(n^c)$, allora $\# = \max\{n, k, a_1,\dots, a_n\}=O(n^c)$. La soluzione basata su programmazione dinamica
ha complessit\`a $O(nk)=O(n^{c+1})$, quindi in $\mathbb{P}$, pertanto SUBSET-SUM non \`e fortemente $\mathbb{NP}$-completo. 
\subsection{Algoritmi pseudo-polinomiali}
Un algoritmo che risolve un problema $R$ per qualsiasi dato $I$ d'ingresso in tempo $P(\#, d)$, con $p$ funzione polinomiale in $\#$ e $d$ ha complessit\`a pseudo-polinomiale. Si noti
come gli algoritmi per SUBSET-SUM basati su programmazione dinamica e memoization sono pseudo-polinomiali. 
\subsubsection{Teorema}
Nessun problema fortemente $\mathbb{NP}$-completo pu\`o essere risolto da un algoritmo pseudo-polinomiale a meno che non sia $\mathbb{P}=\mathbb{NP}$.
\subsection{ESEMPI}
\subsubsection{Partizione (PARTITION)}
Dato un vettore $A$ contenente $n$ interi positivi verificare se esiste un sottoinsieme $S\subseteq\{1\dots n\}$ tale che $\sum\limits_{i\in S}A[i] = \sum\limits_{i\not\in S}A[i]$. Si
noti come questo problema \`e debolmente $\mathbb{NP}$-completo in quanto \`e possibile ridurlo a SUBSET-SUM scegliendo come valore $k$ la met\`a di tutti i valori presenti.
\subsubsection{3-Partizione (3-PARTITION)}
Dati $3n$ interi $\{a_1, \dots, a_{3n}\}$ verificare se esiste una partizione in $n$ triple $T_1, \dots, T_n$ tale che la somma dei tre elementi di ogni $T_j$ \`e la stessa per $1\le j
\le n$. Si noti come questo problema non \`e debolmente $\mathbb{NP}$-completo in quanto non esiste un algoritmo pseudo-polinomiale per risolverlo.
\section{Algoritmi di approssimazione}
Si noti come i problemi pi\`u interessanti sono in forma di ottimizzazione e se il problema di decisione \`e $\mathbb{NP}$-completo nono sono noti algoritmi polinomiali che trovano
soluzioni ammissibili pi\`u o meno vicine a quella ottima. Se \`e possibile dimostrare un limite superiore o inferiore al rapporto fra la soluzione trovata e la soluzione ottima allora
tali algoritmi vengono detti algoritmi di approssimazione. 
\paragraph{Definizione}
Dato un problema di ottimizzazione con funzione di costo non negativa $c$, un algoritmo si dice di $\alpha(n)$-approssimazione se fornisce una soluzione ammissibile $x$ il cui costo
$c(x)$ non si discosta dal costo $c(x*)$ della soluzione ottima $x*$ per pi\`u di un fattore $\alpha(n)$ per un qualunque input di dimensione $n$: 
\begin{align*}
	c(x*)\le c(x)\le\alpha(n)c(x*) & \alpha(n)>1 & \text{Minimizzazione}\\
	\alpha(n)c(x*)\le c(x)\le c(x*) & \alpha(n)<1 & \text{Massimizzazione}
\end{align*}
$\alpha(n)$ pu\`o essere una costante valida per tutti gli $n$. Identificare un valore $\alpha(n)$ e dimostrare che l'algoritmo lo rispetta \`e ci\`o che rende un buon algoritmo un
algoritmo di approssimazione. 
