\chapter{Grafi}
\subsection{Definizioni}
\subparagraph{Grafo orientato}
Si dice grafo orientato la coppia $G=(V, E)$, dove $V$ \`e l'insieme di nodi o vertici ed $E$ \`e l'insieme di coppie ordinate $(u, v)$ di nodi dette anche
archi. 
\subparagraph{Grafo non orientato}
Si dice grafo non orientato la coppia $G=(V, E)$, dove $V$ \`e l'insieme di nodi o vertici ed $E$ \`e l'insieme di coppie non ordinate $(u, v)$ di nodi 
dette anche archi. 
\begin{itemize}
\item Un vertice $v$ \`e detto adiacente a $u$ se esiste un arco $(v, u)$.
\item Un arco $(v, u)$ \`e detto incidente a $v$ e $u$.
\item In un grafo non orientato la relazione di adiacenza \`e simmetrica.
\item $n=|V|$ numero di nodi.
\item $m=|E|$ numero di archi.
\item In un grafo non orientato $m\le\dfrac{n(n-1)}{2}=O(n^2)$.
\item In un grafo orientato $m\le n^2-n=O(n^2)$.
\item La complessit\`a di algoritmi sui grafi deve essere espressa sia un termini di $n$ che di $m$. 
\item Un grafo con un arco tra tutte le coppie di nodi \`e detto completo.
\item Un grafo si dice sparso se ha "pochi archi", ovvero con $m=O(n)$ o $m=O(n\log n)$.
\item Un grafo si dice dento se ha "molti archi", ovvero con $m=\Omega(n^2)$.
\item Un albero libero \`e un grafo connesso con $m=n-1$.
\item Un albero radicato \`e un albero libero sul quale \`e stata scelta una radice. 
\item Un insieme di alberi \`e un grafo detto foresta.
\item Il grado di un nodo \`e il numero di archi incidenti su di esso.
\item Nei grafi orientati si divide il grado in grado entrante per gli archi che incidono su di esso e in grado uscente per gli archi che incidono da esso.
\item Si dice cammino $C$ di lunghezza $k$ in un grafo $G=(V,E)$ una sequenza di nodi $u_0, \dots, u_k$ tali che $(u_i, u_{i+1})\in E$ per $0\le i\le k-1$.
\item Si dice semplice un cammino i cui nodi sono tutti distinti.
\item Si dice peso di un arco un valore associato all'arco dato da una funzione di peso $w: V\times V\rightarrow\mathbb{R}$.
\item Un nodo $v$ si dice raggiungibile da un nodo $u$ se esiste almeno un cammino da $u$ a $v$. 
\item Un grafo non orientato $G=(V, E)$ \`e connesso se e solo se ogni suo nodo \`e raggiungibile da ogni altro nodo. 
\item $G'$ \`e detto sottografo di $G$ $G'\subseteq G$ se e solo se $V'\subseteq V$ e $E'\subseteq E$.
\item $G'$ \`e sottografo massimale di $G$ se e solo se non esiste un altro sottografo $G''$ tale che $G''$ \`e connesso e pi\`u grande di $G'$.
\item Un grafo $G'=(V', E')$ \`e una componente connessa di $G$ se e solo se \`e un sottografo connesso e massimale di $G$.
\item In un grafo non orientato $G=(V, E)$ un ciclo $C$ di lunghezza $k>2$ \`e una sequenza di nodi $u_0, \dots, u_k$ tale che $(u_i, i_{i+1})\in E$ per 
ogni $0\le i\le k-1$ e $u_0=u_k$. Un ciclo \`e detto semplice se tutti i suoi nodi sono distinti. 
\item Un grafo \`e detto aciclico se non contiene cicli. Se orientato \`e detto DAG (directed acyclic graph). 
\item Dato un DAG, un suo ordinamento topologico \`e un ordinamento lineare dei suoi nodi tale che se $(u, v)\in E$, allora $u$ appare prima di $v$ 
nell'odinamento.
\item Un grafo orientato $G=(V, E)$ si dice fortemente connesso se e solo se ogni suo nodo \`e raggiungibile da ogni altro suo nodo.
\item Un grafo $G'=(V', E')$ \`e una componente fortemente connessa di $G$ se e solo se \`e un sottografo connesso e massimale di $G$. 
\end{itemize}
\subsection{Specifica}
\begin{algorithm}
\DontPrintSemicolon
\SetKwComment{comment}{$\%$}{}
\SetKw{Int}{int}
\SetKwData{Set}{Set}
\SetKwData{Node}{Node}
\SetKwFunction{GraphCos}{Graph}
\SetKwFunction{V}{V}
\SetKwFunction{Size}{size}
\SetKwFunction{Adj}{adj}
\SetKwFunction{InsertNode}{insertNode}
\SetKwFunction{InsertEdge}{insertEdge}
\SetKwFunction{DeleteNode}{deleteNode}
\SetKwFunction{DeleteEdge}{deleteEdge}
\caption{Graph}

\GraphCos{}	\comment{Crea un nuovo grafo}
\Set \V{}	\comment{Restituisce l'insieme di tutti i nodi}
\Int \Size{} \comment{Restituisce il numero di nodi}
\Set \Adj{\Node u} \comment{Restituisce i nodi adiacenti al nodo u}
\InsertNode{\Node u} \comment{Inserisce il nodo u nel grafo}
\InsertEdge{\Node u, \Node v}	\comment{Inserisce l'arco tra u e v nel grafo}
\DeleteNode{\Node u}	\comment{Elimina il nodo u dal grafo}
\DeleteEdge{\Node u, \Node v}	\comment{Elimina l'arco tra i nodi u e v dal grafo}
\end{algorithm}

In alcuni casi il grafo \`e dinamico ma sono possibili solo inserimenti e non \`e necessaria la parte di eliminazione (grafo caricato all'inizio).
\section{Memorizzare grafi}
\subsection{Matrice di adiacenza}
Le matrici di adiacenza sono ideali per i grafi densi. Un grafo orientato viene salvato in una matrice di dimensione $n\times n$ bit tale che:
\begin{equation*}
m_{uv}=\begin{cases}1\qquad& (u, v)\in E\\
0 & (u, v)\not\in E
\end{cases}
\end{equation*}
Per il grafo non orientato basta salvare la matrice sopra la diagonale principale esclusa in modo da occupare solamente $\dfrac{n(n-1)}{2}$ bit.
\subparagraph{Grafi pesati}
Per i grafi pesati la matrice di booleani diventa una matrice di reali tale che: 
\begin{equation*}
m_{uv}=\begin{cases}peso\qquad& (u, v)\in E\\
+\infty & (u, v)\not\in E
\end{cases}
\end{equation*}
\subsubsection{Analisi complessit\`a}
\begin{itemize}
\item Spazio richiesto $O(n^2)$.
\item Verificare se $u$ \`e adiacente a $v$ richiede $O(1)$.
\item Iterare su tutti gli archi richiede $O(n^2)$
\end{itemize}
\subsection{Liste di adiacenza}
Le liste di adiacenza sono ideali per i grafi sparsi. Viene creato un vettore di dimensione $n$ tale che in posizione $u$ si trovi un insieme tale che 
contenga gli adiacenti. Occupa $an+bm$ bit. Dove $a$ \`e la dimensione del puntatore al nodo adiacente e $b$ \`e la dimensione dell'adiacente. Per il grafo 
non orientato viene salvato anche l'arco inverso e occupa $an+2bm$.
\subparagraph{Grafi pesati}
Gli elementi della lista puntata sono coppie nodo di arrivo-peso.
\subsubsection{Analisi complessit\`a}
\begin{itemize}
\item Spazio richiesto $O(n+m)$.
\item Verificare se $u$ \`e adiacente a $v$ richiede $O(n)$.
\item Iterare su tutti gli archi richiede $O(n+m)$
\end{itemize}
\subsection{Iterazioni su nodi e archi}
\begin{multicols}{2}
\begin{algorithm}[H]
\SetKwFunction{V}{V}
\caption{Iterazione su tutti i nodi del grafo}
\ForEach{u $\in$ G.\V{}}{
	Esegui operazioni sul nodo u
}
\end{algorithm}

\columnbreak
\begin{algorithm}[H]
\SetKwFunction{V}{V}
\SetKwFunction{Adj}{adj}
\caption{Iterazione su tutti i nodi e archi del grafo}
\ForEach{u $\in$ G.\V{}}{
	Esegui operazioni sul nodo u\;
	\ForEach{v $\in$ G.\Adj{u}}{
		Esegui operazioni sull'arco (u, v)	
	}
}
\end{algorithm}

\end{multicols}
\section{Visite dei grafi}
Dato un grafo $G=(V,E)$ e $r\in V$ radice o sorgente si intende per visita visitare una e una sola volta tutti i nodi che possono essere raggiunti da $r$.
\begin{algorithm}
\DontPrintSemicolon
\SetKwComment{comment}{$\%$}{}
\SetKw{Int}{int}
\SetKwData{Graph}{Graph}
\SetKwData{Set}{Set}
\SetKwData{Node}{Node}
\SetKwFunction{GraphCos}{Graph}
\SetKwFunction{SetCos}{Set}
\SetKwFunction{V}{V}
\SetKwFunction{Size}{size}
\SetKwFunction{Adj}{adj}
\SetKwFunction{InsertNode}{insertNode}
\SetKwFunction{InsertEdge}{insertEdge}
\SetKwFunction{DeleteNode}{deleteNode}
\SetKwFunction{DeleteEdge}{deleteEdge}
\SetKwFunction{GraphTraversal}{graphTraversal}
\SetKwFunction{Insert}{insert}
\SetKwFunction{Remove}{remove}
\caption{\protect\GraphTraversal{\protect\Graph G, \protect\Node r}}

\Set S = \SetCos{} \comment{Insieme generico}
S.\Insert{r}	\comment{Da specificare}
$\{$Marca il nodo r$\}$\;
\While{S.\Size{} $>$ 0}{
	\Node u = S.\Remove{}	\comment{Da specificare}
	$\{$Visita il nodo u$\}$\;
	\ForEach{v $\in$ G.\Adj{u}}{
		$\{$Visita l'arco (u,v)$\}$\;
		\If{v non \`e stato marcato}{
			Marca il nodo v\;
			S.\Insert{v}	\comment{Da specificare}
		}
	}
}


\end{algorithm}

\subsection{Visita in ampiezza o breadth-first search}
In questo tipo di visita si visitano i nodi per livelli crescenti a partire dalla sorgente. Gli obiettivi di questa visita sono di visitare i nodi a 
distanza $k$ prima dei nodi a distanza $k+1$, calcolare il cammino pi\`u breve da $r$ a tutti gli altri nodi (distanze misurate come numero di archi 
attraversati). Generare un albero breadth-first contenente tutti i nodi raggiungibili da $r$ tale per cui il cammino dalla radice $r$ al nodo $u$ 
nell'albero corrisponde al cammino pi\`u breve nel grafo.
\begin{algorithm}[H]
\DontPrintSemicolon
\SetKwComment{comment}{$\%$}{}
\SetKw{Int}{int}
\SetKw{Boolean}{boolean}
\SetKw{New}{new}
\SetKw{True}{true}
\SetKw{False}{false}
\SetKw{Not}{not}
\SetKwData{Graph}{Graph}
\SetKwData{Set}{Queue}
\SetKwData{Node}{Node}
\SetKwFunction{GraphCos}{Graph}
\SetKwFunction{SetCos}{Queue}
\SetKwFunction{V}{V}
\SetKwFunction{Size}{size}
\SetKwFunction{Adj}{adj}
\SetKwFunction{InsertNode}{insertNode}
\SetKwFunction{InsertEdge}{insertEdge}
\SetKwFunction{DeleteNode}{deleteNode}
\SetKwFunction{DeleteEdge}{deleteEdge}
\SetKwFunction{GraphTraversal}{bfs}
\SetKwFunction{Insert}{enqueue}
\SetKwFunction{Remove}{dequeue}
\SetKwFunction{Empty}{empty}
\caption{\protect\GraphTraversal{\protect\Graph G, \protect\Node r}}

\Set Q = \SetCos{} \;
Q.\Insert{r}\;
\Boolean[] visited = \New \Boolean[G.\Size{}]\;
\ForEach{u $\in$ G.\V{} - $\{$r$\}$}{
	visited[u] = \False\;
}
visited[r] = \True\;
\While{\Not Q.\Empty{}}{
	\Node u = Q.\Remove{}\;
	$\{$Visita il nodo u$\}$\;
	\ForEach{v $\in$ G.\Adj{u}}{
		$\{$Visita l'arco (u,v)$\}$\;
		\If{\Not visited[v]}{
			visited[v] = \True\;
			Q.\Insert{v}\;
		}
	}
}
\end{algorithm}

\subsubsection{Calcolo dei cammini pi\`u brevi o del numero di Erd\"os}
L'algoritmo di erdos oltre a determinare la lunghezza del cammino pi\`u breve genera l'arco di copertura bfs che contiene tali cammini.\\
\begin{algorithm}[H]
\DontPrintSemicolon
\SetKwComment{comment}{$\%$}{}
\SetKw{Int}{int}
\SetKw{Boolean}{boolean}
\SetKw{New}{new}
\SetKw{Nil}{nil}
\SetKw{True}{true}
\SetKw{False}{false}
\SetKw{Not}{not}
\SetKwData{Graph}{Graph}
\SetKwData{Set}{Queue}
\SetKwData{Node}{Node}
\SetKwFunction{GraphCos}{Graph}
\SetKwFunction{SetCos}{Queue}
\SetKwFunction{V}{V}
\SetKwFunction{Size}{size}
\SetKwFunction{Adj}{adj}
\SetKwFunction{InsertNode}{insertNode}
\SetKwFunction{InsertEdge}{insertEdge}
\SetKwFunction{DeleteNode}{deleteNode}
\SetKwFunction{DeleteEdge}{deleteEdge}
\SetKwFunction{GraphTraversal}{erdos}
\SetKwFunction{Insert}{enqueue}
\SetKwFunction{Remove}{dequeue}
\SetKwFunction{Empty}{empty}
\caption{\protect\GraphTraversal{\protect\Graph G, \protect\Node r, \protect\Int[] erdos, \protect\Node[] parents}}

\Set Q = \SetCos{}\; 
Q.\Insert{r}\;
\ForEach{u $\in$ G.\V{} - $\{$r$\}$}{
	erdos[u] = $\infty$\;
}
erdos[r] = 0\;
parents[r] = \Nil\;
\While{\Not Q.\Empty{}}{
	\Node u = Q.\Remove{}\;
	$\{$Visita il nodo u$\}$\;
	\ForEach{v $\in$ G.\Adj{u}}{
		\If{erdos[v] = $\infty$}{
			erdos[v] = erdos[u] + 1\;
			parents[v] = u\;			
			Q.\Insert{v}\;
		}
	}
}
\end{algorithm}

\begin{algorithm}[H]
\DontPrintSemicolon
\SetKwComment{comment}{$\%$}{}
\SetKw{Int}{int}
\SetKw{Boolean}{boolean}
\SetKw{New}{new}
\SetKw{Nil}{nil}
\SetKw{True}{true}
\SetKw{False}{false}
\SetKw{Not}{not}
\SetKwData{Graph}{Graph}
\SetKwData{Set}{Queue}
\SetKwData{Node}{Node}
\SetKwFunction{GraphCos}{Graph}
\SetKwFunction{SetCos}{Queue}
\SetKwFunction{V}{V}
\SetKwFunction{Size}{size}
\SetKwFunction{Adj}{adj}
\SetKwFunction{InsertNode}{insertNode}
\SetKwFunction{InsertEdge}{insertEdge}
\SetKwFunction{DeleteNode}{deleteNode}
\SetKwFunction{DeleteEdge}{deleteEdge}
\SetKwFunction{GraphTraversal}{erdos}
\SetKwFunction{Insert}{enqueue}
\SetKwFunction{Remove}{dequeue}
\SetKwFunction{Empty}{empty}
\caption{\protect\GraphTraversal{\protect\Graph G, \protect\Node r, \protect\Int[] erdos, \protect\Node[] parents}}

\Set Q = \SetCos{}\; 
Q.\Insert{r}\;
\ForEach{u $\in$ G.\V{} - $\{$r$\}$}{
	erdos[u] = $\infty$\;
}
erdos[r] = 0\;
parents[r] = \Nil\;
\While{\Not Q.\Empty{}}{
	\Node u = Q.\Remove{}\;
	$\{$Visita il nodo u$\}$\;
	\ForEach{v $\in$ G.\Adj{u}}{
		\If{erdos[v] = $\infty$}{
			erdos[v] = erdos[u] + 1\;
			parents[v] = u\;			
			Q.\Insert{v}\;
		}
	}
}
\end{algorithm}

\begin{algorithm}
\DontPrintSemicolon
\SetKwComment{comment}{$\%$}{}
\SetKw{Int}{int}
\SetKw{Boolean}{boolean}
\SetKw{New}{new}
\SetKw{Nil}{nil}
\SetKw{True}{true}
\SetKw{False}{false}
\SetKw{Not}{not}
\SetKw{Print}{print}
\SetKwData{Graph}{Graph}
\SetKwData{Set}{Queue}
\SetKwData{Node}{Node}
\SetKwFunction{GraphCos}{Graph}
\SetKwFunction{SetCos}{Queue}
\SetKwFunction{V}{V}
\SetKwFunction{Size}{size}
\SetKwFunction{Adj}{adj}
\SetKwFunction{InsertNode}{insertNode}
\SetKwFunction{InsertEdge}{insertEdge}
\SetKwFunction{DeleteNode}{deleteNode}
\SetKwFunction{DeleteEdge}{deleteEdge}
\SetKwFunction{GraphTraversal}{printPath}
\SetKwFunction{Insert}{enqueue}
\SetKwFunction{Remove}{dequeue}
\SetKwFunction{Empty}{empty}
\caption{\protect\GraphTraversal{\protect\Node r, \protect\Node s, \protect\Node[] parents}}

\uIf{r = s}{
	\Print s\;
}
\uElseIf{parents[s] == \Nil}{
	\Print "error"\;
}
\Else{
	\GraphTraversal{r, parents[s], parents}\;
	\Print s\;
}
\end{algorithm}

\subsubsection{Complessit\`a}
La complessit\`a di una bfs \`e $O(m+n)$ in quanto ognuno degli $n$ nodi viene salvato nella coda al pi\`u una volta. Ogni volta che un nodo viene estratto
tutti i suoi archi vengono analizzati una sola volta. Il numero di archi analizzati \`e pertanto $m=\sum\limits_{u\in V}out_d(u)$, dove $out_d(u)$ \`e 
l'out-degree del nodo $u$. 
\subsection{Visita in profondit\`a o depth-first search}
La visita in profondit\`a \`e una visita ricorsiva: per ogni nodo adiacente si visita ricorsivamente tale nodo, visitando ricorsivamente i suoi adiacenti
e cos\`i via. Questo tipo di ricerca \`e spesso una subroutine in altri problemi, visita tutto il grafo, nono solo i nodi raggiungibili dalla radice e
genera pertanto una foresta formata da una collezione di alberi depth-first. Si realizza attraverso uno stack che viene reso implicito attraverso la 
ricorsione.
\subsubsection{Implementazione ricorsiva}
\begin{algorithm}[H]
\DontPrintSemicolon
\SetKwComment{comment}{$\%$}{}
\SetKw{Int}{int}
\SetKw{Boolean}{boolean}
\SetKw{New}{new}
\SetKw{True}{true}
\SetKw{False}{false}
\SetKw{Not}{not}
\SetKwData{Graph}{Graph}
\SetKwData{Set}{Queue}
\SetKwData{Node}{Node}
\SetKwFunction{GraphCos}{Graph}
\SetKwFunction{SetCos}{Queue}
\SetKwFunction{V}{V}
\SetKwFunction{Size}{size}
\SetKwFunction{Adj}{adj}
\SetKwFunction{InsertNode}{insertNode}
\SetKwFunction{InsertEdge}{insertEdge}
\SetKwFunction{DeleteNode}{deleteNode}
\SetKwFunction{DeleteEdge}{deleteEdge}
\SetKwFunction{GraphTraversal}{dfs}
\SetKwFunction{Insert}{enqueue}
\SetKwFunction{Remove}{dequeue}
\SetKwFunction{Empty}{empty}
\caption{\protect\GraphTraversal{\protect\Graph G, \protect\Node u, \protect\Boolean[] visited}}
visited[u] = \True\;
$\{$visita il nodo u in pre-order$\}$\;
\ForEach{v $\in$ G.\Adj{u}}{
	\If{\Not visited[v]}{
		$\{$Visita l'arco (u,v)$\}$\;
		\GraphTraversal{G, v, visited}\;
	}
}
$\{$visita il nodo u in post-order$\}$\;
\end{algorithm}

\subsubsection{Implementazione iterativa}
Una dfs ricorsiva in casi di grafi molto profondi pu\`o superare la dimensione dello stack del linguaggio, si rende pertanto esplicito lo stack utilizzando 
una versione iterativa. Un nodo pu\`o essere inserito nella pila pi\`u volte in quanto il controllo viene fatto all'estrazione. Ha complessit\`a $O(m+n)$ 
con $O(m)$ visite degli archi, $O(m)$ estrazioni e inserimenti e $O(n)$ visite dei nodi. Per la visita in post order quando un nodo viene scoperto viene
inserito nello stack con il tag \emph{discovery}, quando viene estratto un nodo con il tag \emph{discovery} viene re-inserito con il tag \emph{finish} e 
vengono inseriti tutti i suoi vicini. Quando viene estratto un nodo con il tag \emph{finish} viene fatta la post-visita.
\begin{algorithm}
\DontPrintSemicolon
\SetKwComment{comment}{$\%$}{}
\SetKw{Int}{int}
\SetKw{Boolean}{boolean}
\SetKw{New}{new}
\SetKw{True}{true}
\SetKw{False}{false}
\SetKw{Not}{not}
\SetKwData{Graph}{Graph}
\SetKwData{Set}{Stack}
\SetKwData{Node}{Node}
\SetKwFunction{GraphCos}{Graph}
\SetKwFunction{SetCos}{Stack}
\SetKwFunction{V}{V}
\SetKwFunction{Size}{size}
\SetKwFunction{Adj}{adj}
\SetKwFunction{InsertNode}{insertNode}
\SetKwFunction{InsertEdge}{insertEdge}
\SetKwFunction{DeleteNode}{deleteNode}
\SetKwFunction{DeleteEdge}{deleteEdge}
\SetKwFunction{GraphTraversal}{dfs}
\SetKwFunction{Insert}{push}
\SetKwFunction{Remove}{pop}
\SetKwFunction{Empty}{empty}
\caption{\protect\GraphTraversal{\protect\Graph G, \protect\Node r}}

\Set S = \SetCos{} \;
S.\Insert{r}\;
\Boolean[] visited = \New \Boolean[G.\Size{}]\;
\ForEach{u $\in$ G.\V{}}{
	visited[u] = \False\;
}

\While{\Not S.\Empty{}}{
	\Node u = S.\Remove{}\;
	\If{\Not visited[u]}{
		$\{$Visita il nodo u in pre-order$\}$\;
		visited[u] = \True\;
		\ForEach{v $\in$ G.\Adj{u}}{
			$\{$Visita l'arco (u,v)$\}$\;
			S.\Insert{v}\;	
	}	
	}
}
\end{algorithm}

\newpage
\subsubsection{Componenti connesse}
Ci si pone il problema di verificare se il grafo \`e connesso e se non lo \`e di identificare le sue componenti connesse. Un grafo risulta connesso se al
termine della dfs tutti i suoi nodi sono marcati, altrimenti la visita deve ricominciare da capo da un nodo non marcato identificando una nuova componente
del grafo. L'appartenenza ad una componente connessa viene identificata attraverso un vettore \emph{id[]} tale per cui \emph{id[u]} \`e l'identificatore
della componente connessa di appartenenza.
\begin{multicols}{2}
\begin{algorithm}[H]
\DontPrintSemicolon
\SetKwComment{comment}{$\%$}{}
\SetKw{Int}{int}
\SetKw{Boolean}{boolean}
\SetKw{New}{new}
\SetKw{True}{true}
\SetKw{False}{false}
\SetKw{Not}{not}
\SetKwData{Graph}{Graph}
\SetKwData{Set}{Stack}
\SetKwData{Node}{Node}
\SetKwFunction{GraphCos}{Graph}
\SetKwFunction{SetCos}{Stack}
\SetKwFunction{V}{V}
\SetKwFunction{Size}{size}
\SetKwFunction{Adj}{adj}
\SetKwFunction{InsertNode}{insertNode}
\SetKwFunction{InsertEdge}{insertEdge}
\SetKwFunction{DeleteNode}{deleteNode}
\SetKwFunction{DeleteEdge}{deleteEdge}
\SetKwFunction{GraphTraversal}{dfs}
\SetKwFunction{Insert}{push}
\SetKwFunction{Remove}{pop}
\SetKwFunction{Empty}{empty}
\SetKwFunction{CC}{cc}
\SetKwFunction{CCdfs}{ccdfs}

\caption{\protect\Int[] \protect\CC{\protect\Graph G}}

\Int[] id = \New \Int[G.\Size{}]\;
\ForEach{u $\in$ G.\V{}}{
	id[u] = 0\;
}
\Int counter = 0\;
\ForEach{u $\in$ G.\V{}}{
	\If{id[u] == 0}{
		counter += 1\;
		\CCdfs{G, counter, u, id}\;	
	}
}
\Return id\;
\end{algorithm}

\columnbreak
\begin{algorithm}[H]
\DontPrintSemicolon
\SetKwComment{comment}{$\%$}{}
\SetKw{Int}{int}
\SetKw{Boolean}{boolean}
\SetKw{New}{new}
\SetKw{True}{true}
\SetKw{False}{false}
\SetKw{Not}{not}
\SetKwData{Graph}{Graph}
\SetKwData{Set}{Stack}
\SetKwData{Node}{Node}
\SetKwFunction{GraphCos}{Graph}
\SetKwFunction{SetCos}{Stack}
\SetKwFunction{V}{V}
\SetKwFunction{Size}{size}
\SetKwFunction{Adj}{adj}
\SetKwFunction{InsertNode}{insertNode}
\SetKwFunction{InsertEdge}{insertEdge}
\SetKwFunction{DeleteNode}{deleteNode}
\SetKwFunction{DeleteEdge}{deleteEdge}
\SetKwFunction{GraphTraversal}{dfs}
\SetKwFunction{Insert}{push}
\SetKwFunction{Remove}{pop}
\SetKwFunction{Empty}{empty}
\SetKwFunction{CC}{cc}
\SetKwFunction{CCdfs}{ccdfs}

\caption{\protect\Int[] \protect\CCdfs{\protect\Graph G, \protect\Int counter, \protect\Node u, \protect\Int[] id}}

id[u] = counter\;
\ForEach{v $\in$ G.\Adj{u}}{
	\If{id[v] == 0}{
		\CCdfs{G, counter, v, id}\;	
	}
}
\end{algorithm}

\end{multicols}
\subsubsection{Grafo non orientato aciclico}
\begin{algorithm}
\DontPrintSemicolon
\SetKwComment{comment}{$\%$}{}
\SetKw{Int}{int}
\SetKw{Boolean}{boolean}
\SetKw{New}{new}
\SetKw{True}{true}
\SetKw{False}{false}
\SetKw{Not}{not}
\SetKw{Return}{return}
\SetKwData{Graph}{Graph}
\SetKwData{Set}{Stack}
\SetKwData{Node}{Node}
\SetKwFunction{GraphCos}{Graph}
\SetKwFunction{SetCos}{Stack}
\SetKwFunction{V}{V}
\SetKwFunction{Size}{size}
\SetKwFunction{Adj}{adj}
\SetKwFunction{InsertNode}{insertNode}
\SetKwFunction{InsertEdge}{insertEdge}
\SetKwFunction{DeleteNode}{deleteNode}
\SetKwFunction{DeleteEdge}{deleteEdge}
\SetKwFunction{GraphTraversal}{dfs}
\SetKwFunction{Insert}{push}
\SetKwFunction{Remove}{pop}
\SetKwFunction{Empty}{empty}
\SetKwFunction{CC}{cc}
\SetKwFunction{CCdfs}{ccdfs}
\SetKwFunction{HasCycle}{hasCycleRec}

\caption{\protect\Boolean[] \protect\HasCycle{\protect\Graph G, \protect\Node u, \protect\Node p, \protect\Boolean[] visited}}

visited[u] = \True\;
\ForEach{v $\in$ G.\Adj{u} - $\{$p$\}$}{
	\uIf{visited[u]}{
		\Return \True\;	
	}
	\ElseIf{\HasCycle{G, v, u, visited}}{
		\Return \True\;	
	}
}
\Return \False\;
\end{algorithm}


\begin{algorithm}[H]
\DontPrintSemicolon
\SetKwComment{comment}{$\%$}{}
\SetKw{Int}{int}
\SetKw{Boolean}{boolean}
\SetKw{New}{new}
\SetKw{True}{true}
\SetKw{False}{false}
\SetKw{Not}{not}
\SetKw{Nil}{nil}
\SetKw{Return}{return}
\SetKwData{Graph}{Graph}
\SetKwData{Set}{Stack}
\SetKwData{Node}{Node}
\SetKwFunction{GraphCos}{Graph}
\SetKwFunction{SetCos}{Stack}
\SetKwFunction{V}{V}
\SetKwFunction{Size}{size}
\SetKwFunction{Adj}{adj}
\SetKwFunction{InsertNode}{insertNode}
\SetKwFunction{InsertEdge}{insertEdge}
\SetKwFunction{DeleteNode}{deleteNode}
\SetKwFunction{DeleteEdge}{deleteEdge}
\SetKwFunction{GraphTraversal}{dfs}
\SetKwFunction{Insert}{push}
\SetKwFunction{Remove}{pop}
\SetKwFunction{Empty}{empty}
\SetKwFunction{CC}{cc}
\SetKwFunction{CCdfs}{ccdfs}
\SetKwFunction{HasCycleRec}{hasCycleRec}
\SetKwFunction{HasCycle}{hasCycle}

\caption{\protect\Boolean[] \protect\HasCycle{\protect\Graph G}}

\Boolean[] visited = \New \Boolean[G.\Size{}]\;
\ForEach{u $\in$ G.\V{}}{
	visited[u] = \False\;
}
\ForEach{u $\in$ G.\V{}}{
	\If{\Not visited[u]}{
		\If{\HasCycleRec{G, u, \Nil, visited}}{
			\Return \True\;	
		}
	}
}
\Return \False\;
\end{algorithm}

\subsubsection{Grafo orientato aciclico}
\paragraph{Classificazione degli archi}
Ogni qual volta si esamina un arco da un nodo marcato ad un nodo non marcato tale arco diventa un arco dell'albero di copertura dfs $T$. Gli archi non 
inclusi nell'albero possono essere divisi in tre categorie:
\begin{itemize}
\item Se $u$ \`e un antenato di $v$ in $T$ $(u, v)$ \`e detto arco in avanti.
\item Se $u$ \`e un discendente di $v$ in $T$ $(u, v)$ \`e detto arco all'indietro.
\item Se i primi due non sussistono viene detto arco di attraversamento. 
\end{itemize}
Indicato con time un contatore, il vettore \emph{dt} il discovery time, o tempo di scoperta di un nodo e \emph{ft} il finish time o tempo di fine di un 
nodo.\\
\begin{algorithm}[H]
\DontPrintSemicolon
\SetKwComment{comment}{$\%$}{}
\SetKw{Int}{int}
\SetKw{Boolean}{boolean}
\SetKw{New}{new}
\SetKw{True}{true}
\SetKw{False}{false}
\SetKw{Not}{not}
\SetKw{And}{and}
\SetKw{Nil}{nil}
\SetKw{Return}{return}
\SetKwData{Graph}{Graph}
\SetKwData{Set}{Stack}
\SetKwData{Node}{Node}
\SetKwFunction{GraphCos}{Graph}
\SetKwFunction{SetCos}{Stack}
\SetKwFunction{V}{V}
\SetKwFunction{Size}{size}
\SetKwFunction{Adj}{adj}
\SetKwFunction{InsertNode}{insertNode}
\SetKwFunction{InsertEdge}{insertEdge}
\SetKwFunction{DeleteNode}{deleteNode}
\SetKwFunction{DeleteEdge}{deleteEdge}
\SetKwFunction{GraphTraversal}{dfs}
\SetKwFunction{Insert}{push}
\SetKwFunction{Remove}{pop}
\SetKwFunction{Empty}{empty}
\SetKwFunction{CC}{cc}
\SetKwFunction{DfsSchema}{dfs-schema}
\SetKwFunction{HasCycleRec}{hasCycleRec}
\SetKwFunction{HasCycle}{hasCycle}

\caption{\protect\DfsSchema{\protect\Graph G, \protect\Node u, \protect\Int$\&$ time, \protect\Int[] dt, \protect\Int[] ft}}
$\{$Visita il nodo u in pre-order$\}$\;
time += 1\;
dt[u] = time\;
\ForEach{v $\in$ G.\Adj{u}}{
	$\{$Visita l'arco $(u, v)$ qualsiasi$\}$\;
	\uIf{dt[v] = 0}{
		$\{$Visita l'arco $(u, v)$ albero$\}$\;
		\DfsSchema{G, v, time, dt, ft}\;
	}
	\uElseIf{dt[u] $>$ dt[v] \And ft[v] = 0}{
		Visita l'arco $(u, v)$ indietro\;	
	}
	\uElseIf{dt[u] $<$ dt[v] \And ft[v] $\neq$ 0}{
		Visita l'arco $(u, v)$ avanti\;	
	}
	\Else{
		Visita l'arco $(u, v)$ attraversamento\;
	}	
}
$\{$Visita il nodo u in post-order$\}$\;
time +=1\;
ft[u] = time\;
\end{algorithm}

Classificare gli archi permette di dimostrare propriet\`a rispetto a questa classificazione e pertanto costruire degli algoritmi migliori. In particolare 
data una visita dfs di un grafo $G$ per ogni coppia di nodi $u, v\in V$ solo una delle seguenti condizioni \`e vera:
\begin{itemize}
\item Gli intervalli $[dt[u], ft[u]]$ e $[dt[v], ft[v]]$ sono non-sovrapposti: allora $u$ e $v$ non sono discendenti l'uno dell'altro nella foresta DF.
\item L'intervallo $[dt[u], ft[u]]$ \`e contenuto in $[dt[v], ft[v]]$, allora $u$ \`e un discendente di $v$ nella foresta DF.
\item L'intervallo $[dt[v], ft[v]]$ \`e contenuto in $[dt[u], ft[u]]$, allora $v$ \`e un discendente di $u$ nella foresta DF.
\end{itemize}
\paragraph{Teorema}
un grafo \`e aciclico se e solo se non esistono archi all'indietro nel grafo.
\subparagraph{Dimostrazione}
Se esiste un ciclo, sia $u$ il primo nodo del ciclo che viene visitato e $(v, u$ un arco del ciclo. Il cammino che connette $v$ a $u$ verr\`a prima o poi
visitato e da $v$ verr\`a scoperto l'arco all'indietro $(v, u)$. Se esiste un arco all'indietro $(u, v)$, dove $v$ \`e un antenato di $u$ allora esiste
un cammino da $v$ a $u$ e un arco da $u$ a $v$, ovvero un ciclo.
\begin{algorithm}
\DontPrintSemicolon
\SetKwComment{comment}{$\%$}{}
\SetKw{Int}{int}
\SetKw{Boolean}{boolean}
\SetKw{New}{new}
\SetKw{True}{true}
\SetKw{False}{false}
\SetKw{Not}{not}
\SetKw{And}{and}
\SetKw{Nil}{nil}
\SetKw{Return}{return}
\SetKwData{Graph}{Graph}
\SetKwData{Set}{Stack}
\SetKwData{Node}{Node}
\SetKwFunction{GraphCos}{Graph}
\SetKwFunction{SetCos}{Stack}
\SetKwFunction{V}{V}
\SetKwFunction{Size}{size}
\SetKwFunction{Adj}{adj}
\SetKwFunction{InsertNode}{insertNode}
\SetKwFunction{InsertEdge}{insertEdge}
\SetKwFunction{DeleteNode}{deleteNode}
\SetKwFunction{DeleteEdge}{deleteEdge}
\SetKwFunction{GraphTraversal}{dfs}
\SetKwFunction{Insert}{push}
\SetKwFunction{Remove}{pop}
\SetKwFunction{Empty}{empty}
\SetKwFunction{CC}{cc}
\SetKwFunction{DfsSchema}{dfs-schema}
\SetKwFunction{HasCycleRec}{hasCycleRec}
\SetKwFunction{HasCycle}{hasCycle}

\caption{\protect\HasCycle{\protect\Graph G, \protect\Node u, \protect\Int$\&$ time, \protect\Int[] dt, \protect\Int[] ft}}

time += 1\;
dt[u] = time\;
\ForEach{v $\in$ G.\Adj{u}}{
	\uIf{dt[v] == 0}{
		\If{\HasCycle{G, v, time, dt, ft}}{
			\Return \True\;		
		}
	}
	\ElseIf{dt[u] $>$ dt[v] \And ft[v] == 0}{
		\Return \True\;	
	}
}
time +=1\;
ft[u] = time\;
\Return \False\;
\end{algorithm}

\subsubsection{Ordinamento topologico}
L'algoritmo per un ordinamento topologico consiste in una dfs in cui l'operazione di visita consiste nell'aggiungere il nodo in testa ad una lista a tempo
di fine (ovvero post-ordine). L'output sar\`a pertanto una sequenza di nodi ordinata per tempo decrescente di fine. In questo modo quando un nodo \`e finito
tutti i suoi discendenti sono stati aggiunti alla lista e aggiungendolo in testa \`e nell'ordine corretto. 
\begin{algorithm}
\DontPrintSemicolon
\SetKwComment{comment}{$\%$}{}
\SetKw{Int}{int}
\SetKw{Boolean}{boolean}
\SetKw{New}{new}
\SetKw{True}{true}
\SetKw{False}{false}
\SetKw{Not}{not}
\SetKw{And}{and}
\SetKw{Nil}{nil}
\SetKw{Return}{return}
\SetKwData{Graph}{Graph}
\SetKwData{Stack}{Stack}
\SetKwData{Node}{Node}
\SetKwFunction{GraphCos}{Graph}
\SetKwFunction{StackCos}{Stack}
\SetKwFunction{V}{V}
\SetKwFunction{Size}{size}
\SetKwFunction{Adj}{adj}
\SetKwFunction{InsertNode}{insertNode}
\SetKwFunction{InsertEdge}{insertEdge}
\SetKwFunction{DeleteNode}{deleteNode}
\SetKwFunction{DeleteEdge}{deleteEdge}
\SetKwFunction{GraphTraversal}{dfs}
\SetKwFunction{Insert}{push}
\SetKwFunction{Remove}{pop}
\SetKwFunction{Empty}{empty}
\SetKwFunction{CC}{cc}
\SetKwFunction{DfsSchema}{dfs-schema}
\SetKwFunction{HasCycleRec}{hasCycleRec}
\SetKwFunction{HasCycle}{hasCycle}
\SetKwFunction{TopSort}{topSort}
\SetKwFunction{TsDfs}{ts-dfs}

\caption{\protect\Stack \protect\TopSort{\protect\Graph G}}

\Stack S = \StackCos{}\;
\Boolean[] visited = \Boolean[G.\Size{}]\;
\ForEach{u $\in$ G.\V{}}{
	visited[u] = \False\;
}
\ForEach{u $\in$ G.\V{}}{
	\If{\Not visited[u]}{
		\TsDfs{G, visited, S}\;
	}
}
\Return S\;
\end{algorithm}

\begin{algorithm}[H]
\DontPrintSemicolon
\SetKwComment{comment}{$\%$}{}
\SetKw{Int}{int}
\SetKw{Boolean}{boolean}
\SetKw{New}{new}
\SetKw{True}{true}
\SetKw{False}{false}
\SetKw{Not}{not}
\SetKw{And}{and}
\SetKw{Nil}{nil}
\SetKw{Return}{return}
\SetKwData{Graph}{Graph}
\SetKwData{Stack}{Stack}
\SetKwData{Node}{Node}
\SetKwFunction{GraphCos}{Graph}
\SetKwFunction{StackCos}{Stack}
\SetKwFunction{Push}{push}
\SetKwFunction{V}{V}
\SetKwFunction{Size}{size}
\SetKwFunction{Adj}{adj}
\SetKwFunction{InsertNode}{insertNode}
\SetKwFunction{InsertEdge}{insertEdge}
\SetKwFunction{DeleteNode}{deleteNode}
\SetKwFunction{DeleteEdge}{deleteEdge}
\SetKwFunction{GraphTraversal}{dfs}
\SetKwFunction{Insert}{push}
\SetKwFunction{Remove}{pop}
\SetKwFunction{Empty}{empty}
\SetKwFunction{CC}{cc}
\SetKwFunction{DfsSchema}{dfs-schema}
\SetKwFunction{HasCycleRec}{hasCycleRec}
\SetKwFunction{HasCycle}{hasCycle}
\SetKwFunction{TopSort}{topSort}
\SetKwFunction{TsDfs}{ts-dfs}

\caption{\protect\Stack \protect\TsDfs{\protect\Graph G, \protect\Node u, \protect\Boolean[] visited, \protect\Stack S}}

visited[u] = \True\;
\ForEach{v $\in$ G.\Adj{u}}{
	\If{\Not visited[v]}{
		\TsDfs{G, v, visited, S}\;	
	}
}
S.\Push{u}\;
\end{algorithm}

\subsubsection{Algoritmo di Kosaraju}
Utilizzato per trovare le componenti fortemente connesse di un grafo orientato. Per farlo effettua una visita dfs al grafo, calcola il grafo trasposto $G_t$
ed esegue una visita dfs sul grafo $G_t$ utilizzando \emph{cc} esaminando i nodi in ordine inverso di tempo di fine della prima visita. Le componenti 
connesse e i relativi alberi DF rappresentano le componenti fortemente connesse di $G$. (Sostituito dall'algoritmo di Tarjan che richiede una sola visita).
\begin{algorithm}
\DontPrintSemicolon
\SetKwComment{comment}{$\%$}{}
\SetKw{Int}{int}
\SetKw{Boolean}{boolean}
\SetKw{New}{new}
\SetKw{True}{true}
\SetKw{False}{false}
\SetKw{Not}{not}
\SetKw{And}{and}
\SetKw{Nil}{nil}
\SetKw{Return}{return}
\SetKwData{Graph}{Graph}
\SetKwData{Stack}{Stack}
\SetKwData{Node}{Node}
\SetKwFunction{GraphCos}{Graph}
\SetKwFunction{StackCos}{Stack}
\SetKwFunction{Push}{push}
\SetKwFunction{V}{V}
\SetKwFunction{Size}{size}
\SetKwFunction{Adj}{adj}
\SetKwFunction{InsertNode}{insertNode}
\SetKwFunction{InsertEdge}{insertEdge}
\SetKwFunction{DeleteNode}{deleteNode}
\SetKwFunction{DeleteEdge}{deleteEdge}
\SetKwFunction{GraphTraversal}{dfs}
\SetKwFunction{Insert}{push}
\SetKwFunction{Remove}{pop}
\SetKwFunction{Empty}{empty}
\SetKwFunction{CC}{cc}
\SetKwFunction{DfsSchema}{dfs-schema}
\SetKwFunction{HasCycleRec}{hasCycleRec}
\SetKwFunction{HasCycle}{hasCycle}
\SetKwFunction{TopSort}{topSort}
\SetKwFunction{TsDfs}{ts-dfs}
\SetKwFunction{Scc}{scc}
\SetKwFunction{Transpose}{transpose}

\caption{\protect\Int[] \protect\Scc{\protect\Graph G}}

\Stack S = \TopSort{G}		\comment{First visit}
\Graph $G^T$ = \Transpose{G}			\comment{Graph transposal}
\Return cc{$G^T$, S}		\comment{Second visit}
\end{algorithm}

Applicando l'algoritmo topologico si \`e sicuri che se un arco $(u, v)$ non appartiene ad un ciclo, $u$ viene listato prima di $v$ nella sequenza ordinata e
gli archi di un ciclo vengono listati in un qualche ordine che \`e ininfluente. Si utilizza \emph{topSort()} per ottenere i nodi in ordine decrescente di 
tempo di fine. Essendo che ogni passo di questo algoritmo richiede $O(n+m)$, la complessit\`a totale \`e $O(m+n)$.
\paragraph{Calcolo del grafo trasposto}
Dato un grafo orientato $G=(V, E)$ il corrispettivo grafo trasposto $G_t=(V, E_t)$ \`e un grafo con gli stessi nodi ma con gli archi orientati nel senso 
opposto: $E_t=\{(u, v)|(v, u)\in E\}$. L'algoritmo per il suo calcolo ha costo $O(m+n)$.
\begin{algorithm}
\DontPrintSemicolon
\SetKwComment{comment}{$\%$}{}
\SetKw{Int}{int}
\SetKw{Boolean}{boolean}
\SetKw{New}{new}
\SetKw{True}{true}
\SetKw{False}{false}
\SetKw{Not}{not}
\SetKw{And}{and}
\SetKw{Nil}{nil}
\SetKw{Return}{return}
\SetKwData{Graph}{Graph}
\SetKwData{Stack}{Stack}
\SetKwData{Node}{Node}
\SetKwFunction{GraphCos}{Graph}
\SetKwFunction{StackCos}{Stack}
\SetKwFunction{Push}{push}
\SetKwFunction{V}{V}
\SetKwFunction{Size}{size}
\SetKwFunction{Adj}{adj}
\SetKwFunction{InsertNode}{insertNode}
\SetKwFunction{InsertEdge}{insertEdge}
\SetKwFunction{DeleteNode}{deleteNode}
\SetKwFunction{DeleteEdge}{deleteEdge}
\SetKwFunction{GraphTraversal}{dfs}
\SetKwFunction{Insert}{push}
\SetKwFunction{Remove}{pop}
\SetKwFunction{Empty}{empty}
\SetKwFunction{CC}{cc}
\SetKwFunction{DfsSchema}{dfs-schema}
\SetKwFunction{HasCycleRec}{hasCycleRec}
\SetKwFunction{HasCycle}{hasCycle}
\SetKwFunction{TopSort}{topSort}
\SetKwFunction{TsDfs}{ts-dfs}
\SetKwFunction{Scc}{scc}
\SetKwFunction{Transpose}{transpose}

\caption{\protect\Int[] \protect\Transpose{\protect\Graph G}}

\Graph $G^T$ = \GraphCos{G}\;
\ForEach{u $\in$ G.\V{}}{
	$G^T$.\InsertNode{u}\;
}
\ForEach{u $\in$ G.\V{}}{
	\ForEach{v $\in$ G.\Adj{u}}{
		$G^T$.\InsertEdge{v, u}
	}
}
\Return $G^T$\;
\end{algorithm}

\newpage
\paragraph{Calcolo delle componenti connesse}
Invece di esaminare i nodi in ordine arbitrario questa versione li esamina nell'ordine LIFO memorizzato nello stack.
\begin{multicols}{2}
\begin{algorithm}[H]
\DontPrintSemicolon
\SetKwComment{comment}{$\%$}{}
\SetKw{Int}{int}
\SetKw{Boolean}{boolean}
\SetKw{New}{new}
\SetKw{True}{true}
\SetKw{False}{false}
\SetKw{Not}{not}
\SetKw{And}{and}
\SetKw{Nil}{nil}
\SetKw{Return}{return}
\SetKwData{Graph}{Graph}
\SetKwData{Stack}{Stack}
\SetKwData{Node}{Node}
\SetKwFunction{GraphCos}{Graph}
\SetKwFunction{StackCos}{Stack}
\SetKwFunction{Push}{push}
\SetKwFunction{Pop}{pop}
\SetKwFunction{Empty}{isEmpty}
\SetKwFunction{V}{V}
\SetKwFunction{Size}{size}
\SetKwFunction{Adj}{adj}
\SetKwFunction{InsertNode}{insertNode}
\SetKwFunction{InsertEdge}{insertEdge}
\SetKwFunction{DeleteNode}{deleteNode}
\SetKwFunction{DeleteEdge}{deleteEdge}
\SetKwFunction{GraphTraversal}{dfs}
\SetKwFunction{Insert}{push}
\SetKwFunction{Remove}{pop}
\SetKwFunction{Empty}{empty}
\SetKwFunction{CC}{cc}
\SetKwFunction{CCdfs}{ccdfs}
\SetKwFunction{DfsSchema}{dfs-schema}
\SetKwFunction{HasCycleRec}{hasCycleRec}
\SetKwFunction{HasCycle}{hasCycle}
\SetKwFunction{TopSort}{topSort}
\SetKwFunction{TsDfs}{ts-dfs}
\SetKwFunction{Scc}{scc}
\SetKwFunction{Transpose}{transpose}

\caption{\protect\Int[] \protect\CC{\protect\Graph G, \protect\Stack S}}

\Int[] id = \New \Int[G.\Size{}]\;
\ForEach{u $\in$ G.\V{}}{
	id[u] = 0\;
}
\Int counter = 0\;
\While{\Not S.\Empty{}}{
	u = S.\Pop{}\;
	\If{id[u] == 0}{
		counter += 1\;
		\CCdfs{G, counter, u, id}\;
	}
}
\Return id\;
\end{algorithm}

\columnbreak
\begin{algorithm}[H]
\DontPrintSemicolon
\SetKwComment{comment}{$\%$}{}
\SetKw{Int}{int}
\SetKw{Boolean}{boolean}
\SetKw{New}{new}
\SetKw{True}{true}
\SetKw{False}{false}
\SetKw{Not}{not}
\SetKw{And}{and}
\SetKw{Nil}{nil}
\SetKw{Return}{return}
\SetKwData{Graph}{Graph}
\SetKwData{Stack}{Stack}
\SetKwData{Node}{Node}
\SetKwFunction{GraphCos}{Graph}
\SetKwFunction{StackCos}{Stack}
\SetKwFunction{Push}{push}
\SetKwFunction{Pop}{pop}
\SetKwFunction{Empty}{isEmpty}
\SetKwFunction{V}{V}
\SetKwFunction{Size}{size}
\SetKwFunction{Adj}{adj}
\SetKwFunction{InsertNode}{insertNode}
\SetKwFunction{InsertEdge}{insertEdge}
\SetKwFunction{DeleteNode}{deleteNode}
\SetKwFunction{DeleteEdge}{deleteEdge}
\SetKwFunction{GraphTraversal}{dfs}
\SetKwFunction{Insert}{push}
\SetKwFunction{Remove}{pop}
\SetKwFunction{Empty}{empty}
\SetKwFunction{CC}{cc}
\SetKwFunction{CCdfs}{ccdfs}
\SetKwFunction{DfsSchema}{dfs-schema}
\SetKwFunction{HasCycleRec}{hasCycleRec}
\SetKwFunction{HasCycle}{hasCycle}
\SetKwFunction{TopSort}{topSort}
\SetKwFunction{TsDfs}{ts-dfs}
\SetKwFunction{Scc}{scc}
\SetKwFunction{Transpose}{transpose}

\caption{\protect\CCdfs{\protect\Graph G, \protect\Int counter, \protect\Node u, \protect\Int[] id}}

id[u] = counter\;
\ForEach{v $\in$ G.\Adj{u}}{
	\If{id[u] == 0}{
		\CCdfs{G, counter, v, id}\;	
	}
}
\end{algorithm}

\end{multicols}
\paragraph{Dimostrazione di correttezza}
Si costruisca il grafo delle componenti $C(G)=(V_C, E_C)$, dove $V_C=\{C_1, \dots, C_k\}$, dove $C_i$ \`e la i-esima SCC di $G$ e $E_C=\{(C_i, C_j)|\exists 
(u_i, u_j)\in E: u_i\in C_i\land u)j\in C_j\}$. Il grafo delle componenti \`e aciclico e inoltre $C(G^T)=[C(G)]^T$. I discovery e finish time del grafo 
delle componenti corrispondono ai corrispettivi del primo nodo visitato in $C$: $dt(C)=\min\{dt(u)|u\in C\}$ e $ft(C)=\max\{ft(u)|u\in C\}$.
\subparagraph{Teorema}
Siano $C$ e $C'$ due distinte SCCs nel grafo orientato $G=(V, E)$. Se c'\`e un arco $(C, C')\in E_C$, allora $ft(C)>ft(C')$.
\subparagraph{Corollario}
Siano $C_u$ e $C_v$ due SCCs distinte nel grafo orientato $G=(V, E)$, se c\`e un arco $(u, v)\in E_T$ tale che $u\in C_u$ e $v\in C_v$, allora $ft(C_u)<
ft(C_v)$. 
\begin{align*}
(u, v)\in E_T & \Rightarrow\\
(v, u)\in E & \Rightarrow\\
(C_v, C_u)\in E_C & \Rightarrow\\
ft(C_v)>ft(C_u) & \Rightarrow\\
ft(C_u)<ft(C_v) &
\end{align*}
\subparagraph{Conclusione}
Se le componenti $C_u$ e $C_v$ sono connesse da un arco $(u, v)\in E_T$ allora dal corollario $ft(C_u)<ft(C_v)$ e dall'algoritmo la visita di $C_v$ 
inizier\`a prima della visita di $C_u$. Non esistono cammini tra $C_u$ e $C_v$ in $E_T$ altrimenti il grafo sarebbe ciclico, pertando dall'algoritmo la
visita di $C_v$ non raggiunger\`a mai $C_u$, pertanto \emph{cc()} assegner\`a correttamente gli identificatori delle componenti ai nodi.
